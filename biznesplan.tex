\documentclass[12pt]{article}

\usepackage[utf8]{inputenc}
\usepackage[T1]{fontenc}
\usepackage[polish]{babel}
\usepackage{polski}
\selectlanguage{polish}

% \usepackage{color}
% \usepackage[backend=biber]{biblatex}
% \usepackage{amsmath}
% \usepackage{amsfonts}
% \usepackage{amssymb}
% \usepackage{hyphenat}
% \usepackage{graphicx}
% \usepackage{csquotes}
% \usepackage[makeroom]{cancel}
% \usepackage{amsmath}
% \addbibresource{bibliography.bib}

\usepackage{vwcol}
\usepackage[letterspace=20]{letterspace}

\DeclareUnicodeCharacter{00A0}{~}

\begin{document}

\begin{titlepage}
    \hspace{0em}
    \vfill
    \begin{vwcol}[widths={0.2,0.8}, sep=2em, rule=0em]
        {\normalsize \textbf{\uppercase{Biznesplan}}}

        \vfill\eject

        {\LARGE \lsstyle{Wirtualna perkusja Mike'a Malyana}} \newline \vspace{1em}
        {\large Tomasz Witczak, 216920} \newline \vspace{1em}
        {\normalsize \today}
    \end{vwcol}
    \vfill
    \hspace{0em}
\end{titlepage}

\section{Streszczenie}

% - [ ] Streszczenie jest zwięzłą, miniaturową wersją całego biznesplanu i stanowi zamkniętą całość
% - [ ] Nie należy go utożsamiać ze wstępem ani wprowadzeniem
% - [ ] Jest to mały biznesplan wewnątrz większego
% - [ ] Streszczenie należy napisać z wielką starannością, ponieważ jego niska jakość może skutecznie zniechęcić potencjalnego inwestora
% - [ ] Należy używać zrozumiałego języka i precyzyjnie wyjaśnić dlaczego właśnie to przedsięwzięcie powinno zostać sfinansowane

% - [ ] Objętość streszczenia to ok. 1-4 strony
% - [ ] Zawiera ono najważniejsze kwestie, kluczowe dane i argumenty mające poinformować czytającego o przedsięwzięciu i przekonać go, że ta firma odniesie sukces
% - [ ] Streszczenie choć umieszczone na początku, jest sporządzane już po przygotowaniu biznesplanu

% - [ ] Powinno być na tyle interesujące dla czytelników, aby chcieli przeczytać resztę dokumentu
% - [ ] Można je rozpocząć od jakiegoś poruszającego wyobraźnię stwierdzenia, które ukazuje zaobserwowaną okazję
% - [ ] Przygotowując jakąś usługę dla osób starszych można zacząć na przykład w ten sposób: Główny Urząd Statystyczny przewiduje, że w ciągu najbliższych 20 lat liczba ludzi w wieku powyżej 80 lat wzrośnie dwukrotnie
% - [ ] Czytający odniesie wrażenie, że jest potencjał do wykorzystania

% - [ ] Streszczenie należy zacząć od zwięzłego opisu pomysłu, który ma zainteresować potencjalnych inwestorów
% - [ ] Powinno zawierać
%   - [ ] nazwę przedsiębiorstwa
%   - [ ] nazwiska właścicieli
%   - [ ] koncepcję rozwoju firmy
%   - [ ] przewidywane koszty i zyski
%   - [ ] plan wykorzystania środków inwestycyjnych

% - [ ] W streszczeniu powinno się znaleźć
%   - [ ] wyjaśnij co firma oferuje, na jakim rynku, z kim konkuruje
%   - [ ] wskaż dlaczego osiągniesz sukces, jakie czynniki o tym zadecydują
%   - [ ] określ jak wysokie prognozujesz zyski, jaki będzie kierunek rozwoju firmy

% - [ ] W streszczeniu biznesplanu powinny znaleźć się tylko to co istotne. Niezbędne minimum to:
%   - [ ] firma
%   - [ ] zespół
%   - [ ] oferta
%   - [ ] czynniki decydujące o sukcesie przedsięwzięcia
%   - [ ] przewidywane zyski
%   - [ ] główne ryzyko i sposoby jego minimalizacji
%   - [ ] oczekiwania w stosunku do czytającego

% - [ ] W streszczeniu trzeba opisać
%   - [ ] siebie
%   - [ ] swój zespół
%   - [ ] historię
%   - [ ] czym się zajmujesz
%   - [ ] co jest takiego w Twojej ofercie
%   - [ ] że odniesiesz sukces
%   - [ ] jaki osiągniesz zysk
%   - [ ] czego oczekujesz od czytającego

% - [ ] Możesz napisać np. o tym
%   - [ ] jak nazywa się firma
%   - [ ] w jaki sposób będzie zarabiać
%   - [ ] jaka jest jej misja
%   - [ ] kim są klienci docelowi
%   - [ ] co im oferuje i czym to się różni od konkurencji
%   - [ ] w jaki sposób zamierzasz sprostać konkurencji
%   - [ ] jak duży jest rynek
%   - [ ] ilu klientów zamierzasz pozyskać w ciągu pierwszych lat
%   - [ ] jaką przyjmiesz strategię sprzedaży i dystrybucji
%   - [ ] jaki masz slogan reklamowy
%   - [ ] jak wygląda struktura przychodów
%   - [ ] jaki jest plan promocji i działań marketingowych
%   - [ ] jakie są cele sprzedażowe firmy i jak je osiągniesz
%   - [ ] jak wyglądają prognozy sprzedaży
%   - [ ] kiedy osiągniesz próg rentowności
%   - [ ] jakie masz źródła finansowania
%   - [ ] jakie będą potrzeby finansowe w najbliższych latach
%   - [ ] z kim będziesz współpracować

% - [ ] Musisz jednak pisać zwięźle i konkretnie
% - [ ] Uważa się, że czytelnik powinien przeczytać i zrozumieć streszczenie w ciągu 5-10 minut
% - [ ] Można to sprawdzić dając je do przeczytania komuś kto nic nie wie o przygotowanej koncepcji i poprosić o krótkie zreferowanie tego co się dowiedział

% - [ ] Streszczenie powinno zawierać odpowiedzi na 5 pytań
%   - [ ] Czy istnieje rynek na danego typu produkty lub usługi?
%   - [ ] Czy przedsiębiorstwo oferuje produkty lub usługi pozwalające zdobyć rynek?
%   - [ ] Czy przedsiębiorstwo dysponuje ludźmi zdolnymi poprowadzić je do sukcesu?
%   - [ ] Jakie zyski przedsiębiorstwo będzie przynosić?
%   - [ ] W jakim kierunku przedsiębiorstwo będzie zmierzać?

\section{Opis produktu}

% - [ ] Firmę prowadzi się po to, aby zarabiać na sprzedaży produktów lub usług
% - [ ] Ich opis jest podstawowym celem biznesplanu
% - [ ] Jego czytelnik chce wiedzieć co planujesz sprzedawać i dlaczego klienci będą chcieli to kupić
% - [ ] Należy zastanowić się jakie potrzeby produkt zaspokaja

% - [ ] Ważny jest innowacyjny produkt lub usługa, którą chcesz wprowadzić na rynek i określenie czym będą się one różniły, a raczej wyróżniały na tle tych oferowanych przez konkurencję
% - [ ] Opisz produkt podając jego rozmiary, kolory, wysokość, wagę funkcje
% - [ ] Należy omówić funkcje jakie spełnia produkt i korzyści uzyskane przez klienta dzięki niemu

% - [ ] Odpowiedz sobie na kilka pytań:
%   - [ ] Kto jest odbiorcą?
%   - [ ] Co oferuję klientowi?
%   - [ ] Jacy kooperanci są mi potrzebni?
%   - [ ] Jakie konkurencyjne produkty są już na rynku?
%   - [ ] Na czym polega innowacyjność mojej oferty?

% - [ ] Zacznij od ogólnej charakterystyki oferty i potem przejdź do szczegółów
% - [ ] Opisz swój produkt lub usługę, a także branżę w jakiej działasz, jakie są koszty produktu
% - [ ] Napisz w jaki sposób go pozyskasz
% - [ ] Jeśli zamierzasz sam go wyprodukować wskaż komponenty, z których jest wytwarzany i określ koszty
% - [ ] Jeśli jest to usługa napisz kto wykonuje konkretne prace i gdzie

% - [ ] Jeśli to ułatwi zrozumienie możesz zawrzeć ilustracje sprzedawanego towaru
% - [ ] Warto też opisać doświadczenie jakim dysponujesz, a które legitymizuje Twoją zdolność do sprzedaży
% - [ ] Wskaż kierunki rozwoju działalności, które są możliwe dla tego produktu/usługi

% - [ ] Produkt powinien zostać opisany możliwie najdokładniej, ponieważ osoby z zewnątrz nie znają jego specyfiki
% - [ ] Dokonując charakterystyki należy uwzględnić następujące informacje:
%   - [ ] rodzaj produktu,
%   - [ ] jego główne cechy i zalety,
%   - [ ] wyjaśnienie czy jest to produkt, który już istnieje, czy jest dopiero projektowany,
%   - [ ] pokazanie czym produkt różni się od innych dostępnych na rynku, na czym polega jego wyjątkowość,
%   - [ ] koszt jednostkowy produktu,
%   - [ ] cena produktu,
%   - [ ] zysk jednostkowy,
%   - [ ] posiadane znaki towarowe,
%   - [ ] sposób zapewnienia jakości produktu,
%   - [ ] wskazanie czy wyrób stwarza możliwość jego modyfikacji czy ulepszeń przy stosunkowo niewielkich zmianach,
%   - [ ] planowana wielkość produkcji (miesięczna, kwartalna, roczna)

% - [ ] Jeśli masz takie dane, napisz również co myślą o produkcie klienci

% - [ ] Co powinno się znaleźć w opisie produktu/usługi
%   - [ ] fizyczny opis produktu (m.in. kolor, rozmiar, waga, funkcje, wykonanie)
%   - [ ] zalety produktu, dlaczego klient będzie chciał go kupić, co go wyróżnia na tle innych
%   - [ ] cena, koszt pozyskania, wielkość produkcji, metody zapewnienia wysokiej jakości

\subsection{Produkt}

Inwestycja dotyczy biblioteki sampli perkusyjnych sygnowanych przez muzyka
Mike'a Malyana. Produkt nie jest pierwszym takim na rynku, jednakże bazując na
sukcesach i porażkach konkurencji, rozwija pomysł i wprowadza nową
funkcjonalność. Dostarczony w formie wtyczek VSTi, zadziała na każdym systemie
operacyjnym i w każdym głównym programie DAW. Użycie sztucznej inteligencji do
humanizacji odsłuchu zapewni klientowi jakość, której próżno szukać u
konkurencji.

\subsection{Charakterystyka produktu}

Biblioteka zawierać będzie nagrania poszczególnych instrumentów, z których
składał się zestaw perkusyjny użyty przez Malyana w 2017 roku na płycie
"Foreword" zespołu Disperse. Kolekcja skierowana jest na prostotę użycia, stąd
znajdą się w niej tylko niezbędne elementy takie jak: werbel, stopa, trzy tomy
i talerze. Każdy instrument zawierać będzie nagrania wykonane wielokrotnie w
różnych konfiguracjach. Uzyskane dzięki temu sample pozwolą na dobre
odwzorowanie dynamiki i artykulacji.

Interfejs do biblioteki zostanie udostępniony w formie programu VST. Taką
wtyczkę będzie można bez problemu zaimportować w większości używanych dzisiaj
stacji muzycznych. Użytkownikowi zostanie przedstawiona uproszczona
wizualizacja perkusji z możliwością dostosowywania opcji za pomocą suwaków.
Dołączone zostaną także gotowe mapowania MIDI umożliwiające dostosowanie
używanych artykulacji pod preferencje klienta. Po załadowaniu instrumentu na
ścieżkę, należy dołączyć plik MIDI z sekwencją nut w celu odegrania dźwięków.

\subsection{Innowacyjność produktu}

Produkt jest jedynym na rynku, który uwzględnia pełen zestaw perkusyjny Mike'a
Malyana. Istotnie, nowatorskie użycie sieci splotowych jako sztucznej
inteligencji zapewnia niespotkaną wcześniej humanizację dźwięków. Algorytm
został nauczony na bazie prawdziwych nagrań Malyana i dzięki temu zapewnia
odwzorowanie niuansów jego gry w programie. Dzięki naszemu produktowi możesz
mieć realnie brzmiące ścieżki perkusyjne na swoim albumie przy niskich
kosztach.

\subsection{Zalety i wady}

Zalety:

- pełen zestaw perkusyjny Malyana
- idealne odwzorowanie niuansów i detali (dynamika, timing i artykulacja)
- wysoka jakość brzmienia
- gotowość ścieżek do miksu/masteringu (minimalna wymagana obróbka)
- sztuczna inteligencja
- niskie zużycie pamięci RAM
- atrakcyjny i nowoczesny, aczkolwiek prosty interfejs użytkownika

Wady:

- nakierunkowanie na niszę konsumencką - idealny klient to muzyk szukający
  nowoczesnego brzmienia perkusji
- ubogość zestawu - jeśli klientowi zależy na dużym wyborze brzmień, ten zestaw
  nie jest dla niego

\subsection{Koszty i komponenty}

Na koszty składać się będzie:

- podpisanie umowy z muzykiem
- zatrudnienie inżyniera oraz wynajęcie studia w Anglii
- proces nagrania i przygotowania sampli
- wytworzenie oprogramowania i nauka sztucznej inteligencji
- marketing internetowy
- wsparcie dla produktu w przypadku wystąpienia błędów (aktualizacje
  oprogramowania)

\section{Zespół zarządzający}

% - [ ] Należy zawrzeć:
%   - [ ] charakterystyki osób zarządzających firmą,
%   - [ ] ich doświadczenie,
%   - [ ] wykształcenie i osiągnięcia zawodowe
% - [ ] Trzeba omówić te kwalifikacje, które są niezbędne dla realizacji biznesplanu
% - [ ] Warto też wspomnieć o doradcach, takich jak: księgowi, firmy PR, itp.

\subsection{Struktura kierownicza firmy}

Wspólnicy i wnoszone wkłady:

- Tomasz Witczak - wkład: pieniądze (95 000 zł)
- Łukasz Ziułkiewicz - wkład: pieniądze (15 000 zł)
- Mike Malyan - wkład: perkusja (40 000 zł)

Struktura organizacyjna:

- dyrektor naczelny - Tomasz Witczak
- dyrektor do spraw marketingu i sprzedaży internetowej - Jan Kowalski
- dyrektor do spraw inżynierii oprogramowania - Michał Rogala
- dyrektor do spraw produkcji muzycznej - Łukasz Ziułkiewicz
- dyrektor do spraw kontroli jakości - Albert Pietrzak

Zadania kierownicze:

- dyrektor naczelny - nadzór nad przebiegiem pełnego procesu od projektu,
  poprzez wykonanie, wdrożenie i wsparcie produktu
- dyrektor do spraw marketingu i sprzedaży internetowej - odpowiada za
  wypromowanie marki w mediach społecznościowych oraz przyciągnięcie klientów
- dyrektor do spraw inżynierii oprogramowania - zajmuje się nadzorem kwestii
  związanych z technologicznym aspektem wtyczki VSTi
- dyrektor do spraw produkcji muzycznej - jako producent muzyczny, zarządza
  początkowym etapem nagrań perkusyjnych oraz ich obróbką
- dyrektor do spraw kontroli jakości - jako doświadczony muzyk, przeprowadza
  proces kontroli jakości oprogramowania pod kątem muzycznym

\section{Rynek i konkurencja}

\subsection{Rynek}

\subsubsection{Wielkość rynku i możliwości jego rozwoju}

% - [ ] Wielkość rynku powinna być podana w liczbach
%   - [ ] liczby klientów
%   - [ ] liczby sprzedanych jednostek
%   - [ ] spodziewanych obrotów
% - [ ] Uwzględniamy przewidywania co do rozwoju rynku, wskazujemy czynniki istotne dla tego rozwoju
% - [ ] W tym celu można sięgnąć do literatury fachowej, baz teleadresowych, badań rynku, stowarzyszeń czy zrzeszeń, Internetu, wywiadów, itp.

\subsubsection{Segmentacja rynku}

% - [ ] Liczbę klientów i ich zachowanie w konkretnym segmencie rynku należy zdefiniować i udokumentować
% - [ ] Celem segmentacji jest wybór grupy docelowej dla produktu
% - [ ] Należy określić przewidywaną wielkość sprzedaży
% - [ ] Segmentacji można dokonać na podstawie
%   - [ ] miejsca zamieszkania
%   - [ ] danych demograficznych
%   - [ ] typowych zachowań
%   - [ ] zamożności

\subsection{Konkurencja}

% - [ ] Określić głównych konkurentów i zbadać ich pozycję na rynku
% - [ ] Kryteria
%   - [ ] wysokość obrotów
%   - [ ] wielkość przychodów ze sprzedaży
%   - [ ] wysokość cen
%   - [ ] udział w rynku
%   - [ ] kanały dystrybucji
%   - [ ] itp.

\section{Plan marketingowy}

% - [ ] Plan marketingowy określa zestaw posunięć firmy w strefie marketingu i pozwala wyrobić sobie pogląd na temat przyjętej strategii
% - [ ] Określa on co, komu, kiedy i jak firma sprzedaje
% - [ ] Jego uzupełnieniem jest plan sprzedaży, który z kolei opisuje w jaki sposób i komu są sprzedawane produkty lub usługi.

% - [ ] Działania marketingowe i sprzedaż produktów są jednymi z najistotniejszych elementów powodzenia całego przedsięwzięcia
% - [ ] Najważniejsze tutaj to wziąć pod uwagę kilka czynników takich jak
%   - [ ] produkt
%   - [ ] cena
%   - [ ] miejsce i sposób dystrybucji
%   - [ ] promocja i reklama
% - [ ] Są one potocznie nazywane mieszanką marketingową. W planie marketingowym należy je scharakteryzować
% - [ ] Często wyznaczają jego strukturę, stanowią kolejno omawiane podpunkty

% - [ ] Przy planowaniu działań trzeba uwzględnić budżet
% - [ ] Dobrze skalkulować na co firmę stać a na co nie
% - [ ] W przypadku niewielkich środków najlepiej skierować przekaz bezpośrednio do grupy docelowej

% - [ ] Właściwie napisany plan określa z jakich narzędzi będziesz korzystać, aby dotrzeć do klientów
% - [ ] Powinno się
%   - [ ] dokładnie wskazać metody reklamy i promocji używane do sprzedaży
%   - [ ] napisać kiedy, gdzie, dlaczego, w jaki sposób zamierza się dotrzeć do klientów i jakie to pociągnie za sobą koszty

% - [ ] Istnieje wiele sposobów reklamy i promocji
% - [ ] Można skorzystać z gazet, broszur, katalogów, telewizji i radia, ulotek reklamowych, informacji prasowych, tablic ogłoszeń
% - [ ] W Internecie wykupić usługę pozycjonowania w wyszukiwarkach, rozsyłać promocyjne e-maile, biuletyny
% - [ ] Wszystkie te narzędzia mogą zostać użyte wspólnie w ramach 4 podstawowych platform marketingowych
%   - [ ] pozycjonowania
%   - [ ] pozyskiwania klientów
%   - [ ] ich utrzymania
%   - [ ] strategii osiągania przychodów

\subsection{Pozycjonowanie firmy w Internecie} % \subsection{Marketing internetowy}

% - [ ] We współczesnym świecie bardzo ważna jest obecność firmy w Internecie
% - [ ] Należy zadbać o to, aby było ją łatwo znaleźć
% - [ ] Należy zastanowić się w jaki sposób dotrzeć do szerokiej rzeszy klientów, którzy korzystają z tego medium

% - [ ] Do pozycjonowania można zatrudnić profesjonalną firmę zewnętrzną
% - [ ] Jeśli nie ma się doświadczenia jest to dobre wyjście, ponieważ nieumiejętne pozycjonowanie może bardziej zaszkodzić niż pomóc

% - [ ] Pozycjonowanie obejmuje m.in. zgłoszenie swojej strony do wyszukiwarek
% - [ ] Obecność w nich i zajmowanie wysokiej pozycji na wyszukiwane słowa kluczowe jest ważna, żeby zapewnić stronie internetowej oglądalność
% - [ ] W tym miejscu możesz wymienić te wyszukiwarki (np. yahoo.com, google.pl, pl.ask.com)

% - [ ] Dopilnuj, aby na materiałach firmowych był zamieszczony adres strony internetowej i adres e-mailowy
% - [ ] Napisz o tym, że np. zapewnisz klientom możliwość zapisania się do newslettera itp.
% - [ ] Określ w jaki sposób będziesz się komunikować z klientami w Internecie
% - [ ] Przemyśl koncepcję swojej strony i co dzięki niej klient będzie mógł zrobić

% - [ ] Opisz brane pod uwagę sposoby reklamy
% - [ ] Zdecyduj gdzie chcesz je wykupić i dlaczego właśnie tam, jaką przybiorą formę

\subsection{Pozyskiwanie klientów}

% - [ ] Pozyskać klientów można w rozmaity sposób, z wykorzystaniem wielu metod
% - [ ] To kiedy planujesz dotrzeć do klientów jest równie ważne jak to jakimi metodami to zrobisz
% - [ ] Jeśli wybierzesz reklamę w radiu, podaj jej długość, częstotliwość ukazywania się
% - [ ] Jeśli zdecydujesz się wziąć udział w targach/ imprezach/ seminariach, napisz kiedy się odbywają, jaką noszą nazwę
% - [ ] Stwórz harmonogram wykupowania reklam

% - [ ] Działania które wyszczególnisz, muszą mieć umocowanie w budżecie, a także przedstawiać realne rezultaty
% - [ ] Napisz gdzie wydasz pieniądze, ile, czemu w ten sposób i czego się spodziewasz po takiej kampanii
% - [ ] Koszty należy wpisać w plan finansowy

% - [ ] Warto również uwzględnić darmowe sposoby reklamy
% - [ ] Z pomocą przyjdzie tutaj Internet
% - [ ] Obecność w portalach społecznościowych (np. Facebook, Twitter) nic nie kosztuje, a może zachęcić klientów do zakupu
% - [ ] Inny sposób to wymiana linków z firmami, które sprzedają towar uzupełniający ofertę firmy (a więc nie ma konfliktu interesów)
% - [ ] Można również zamieszczać artykuły/posty na wielu stronach funkcjonujących w Internecie

% - [ ] Istotne są działania związane z PR
% - [ ] Jeśli produkt lub usługa jest czymś wyjątkowym, co wzbudza ciekawość możesz wysyłać informację prasową do mediów
% - [ ] Wydawcy stron chętnie zamieszczają tego rodzaju newsy

% - [ ] Wybrane metody promocji i reklamy:
%   - [ ] Klasyczna reklama - ogłoszenia w prasie, radiu, telewizji, kinie, Internecie
%   - [ ] Marketing bezpośredni - mailing, rozdawanie ulotek na ulicy
%   - [ ] Public relations - artykuły o produktach firmy napisane przez pracownika lub wynajętego dziennikarza
%   - [ ] Wystawy, targi
%   - [ ] Wizyty u klientów

\subsection{Utrzymywanie klientów}

% - [ ] Utrzymanie klienta jest o wiele tańsze niż zdobycie go
% - [ ] Można to zrobić poprzez np. stworzenie listy klientów zawierającej ich adresy
% - [ ] Następnie zaplanować dla nich jakąś korzyść, utrzymywać z nimi kontakt, pozostać w ich świadomości
% - [ ] Jednym ze sposobów są programy lojalnościowe lub preferencyjne

% - [ ] Innym rozsyłanie biuletynów, newslettera albo utworzenie na swojej stronie internetowej forum, na którym klienci będą wymieniali opinie
% - [ ] To ostatnie jest dość ryzykowne, ponieważ mogą to być opinie negatywne
% - [ ] Dobrze jest w takim przypadku stworzyć strategię reagowania na nie (ignorowanie negatywnych opinii nie jest korzystnym wyjściem).

\subsection{Struktura przychodów}

% - [ ] Celem działalności jest zysk
% - [ ] Wygenerujesz go dzięki sprzedaży towarów lub usług
% - [ ] Opisz w jaki sposób będziesz prowadzić sprzedaż

% - [ ] Podaj strukturę cen i plan uzyskania przychodu
% - [ ] Zdecyduj po jakiej cenie będą sprzedawane produkty lub usługi
% - [ ] Jednostką sprzedaży jest pojedynczy produkt lub usługa, np. godzina pracy, rzecz którą sprzedajesz
% - [ ] Cena jaką ustalisz musi być na tyle wysoka, aby pokryć wydatki i zapewnić zysk, a jednocześnie tak dobrana, aby klient był gotowy ją zapłacić (niekoniecznie oznacza to ceną najniższą)
% - [ ] Dzięki badaniom rynkowym dowiesz się ile powinna wynosić
% - [ ] Nowe firmy preferują najczęściej mniej klientów i wyższą cenę

% - [ ] Jeśli planujesz sprzedawać przez Internet, wyjaśnij jak zamienisz ruch na swojej stronie na przychody
% - [ ] Ważny jest przyjazny użytkownikom mechanizm sprzedaży
% - [ ] Innym sposobem na zarabianie na stronie jest pozyskanie sponsora lub uczestnictwo w programach partnerskich

\subsection{Strategia sprzedaży}

% - [ ] Strategia sprzedaży powinna być zsynchronizowana ze strategią marketingową
% - [ ] Należy opisać jak będzie sprzedawany produkt lub usługa, kto będzie to robił
% - [ ] Jeśli chcesz skorzystać z własnego personelu, a produkt jest czymś skomplikowanym technologicznie, możesz rozważyć przeprowadzenie specjalnych szkoleń związanych z jego obsługą

% - [ ] Sposoby dystrybucji występują pośrednie i bezpośrednie, np.
%   - [ ] Detaliści zewnętrzni
%   - [ ] Przedstawiciele zewnętrzni
%   - [ ] Licencja franchisingowa
%   - [ ] Hurtownicy
%   - [ ] Sklepy firmowe
%   - [ ] Własna sieć sprzedaży bezpośredniej
%   - [ ] Mailing
%   - [ ] Internet

% - [ ] Opisz w jaki sposób sprzedawcy będą utrzymywać kontakt z klientami, jakie umiejętności powinni posiadać, w jaki sposób ich zrekrutujesz
% - [ ] Jeśli sam/a będziesz sprzedawcą, powinieneś/powinnaś określić swoje doświadczenie w sprzedaży
% - [ ] Napisz co wiesz o sprzedawaniu tego produktu, usługi

% - [ ] Uwzględnij w działaniach kontakty telefoniczne i reklamę bezpośrednią, opisz szczegóły w planie
% - [ ] Podaj płace i prowizje sprzedawców, świadczenia i zachęty
% - [ ] Strategię sprzedaży można zaprezentować w formie tabelki, w której będą przedstawione prognozy sprzedaży na kolejne lata

\section{System biznesowy i organizacja}

% - [ ] Należy przemyśleć w jaki sposób dokonamy podziału prac, aby nasze przedsięwzięcie było realizowane najwydajniej
% - [ ] Wybierz działania, jakie będą wykonywane w Twojej firmie i pogrupuje w bloki funkcyjne, np.
%   - [ ] badania i rozwój
%   - [ ] produkcja
%   - [ ] marketing
%   - [ ] sprzedaż
%   - [ ] obsługa/serwis

% - [ ] Działania wykraczające poza główną domenę działalności firmy powinny zostać zlecone podwykonawcom
% - [ ] Działania dla firmy strategiczne należy zostawić pod bezpośrednią kontrolą zarządu firmy, ale np. księgowość, czy zarządzanie kadrami można zlecić komuś z zewnątrz

% - [ ] Niezwykle ważne jest stworzenie prostej i funkcjonalnej struktury organizacyjnej
% - [ ] Należy być przygotowanym do elastycznej reorganizacji firmy, przynajmniej w pierwszych kilku latach działalności
% - [ ] Zdecyduj kto za co odpowiada
% - [ ] Prosta struktura ułatwi sporządzenie opisów stanowisk i zatrudnienia na nie odpowiednich osób

\section{Harmonogram realizacji procesów}

% - [ ] Poszczególne zadania należy pogrupować w pakiety robocze
% - [ ] Każdy z nich musi mieć kolejne etapy, które kończą się osiągnięciem zamierzonego celu
% - [ ] Lepiej planować pesymistycznie niż optymistycznie

\section{Możliwości i zagrożenia}

% - [ ] Źródłem zagrożenia może być sama firma jak też czynniki zewnętrzne
% - [ ] Oceny zagrożeń można dokonać przez stworzenie scenariuszy rozwoju firmy. W biznesplanie powinny być trzy:
%   - [ ] scenariusz bazowy - najbardziej prawdopodobny
%   - [ ] scenariusz optymistyczny
%   - [ ] scenariusz pesymistyczny - wszystkie zagrożenia jakie się pojawią

\section{Plany finansowe}

% - [ ] Zastanów się ile pieniędzy trzeba na założenie i prowadzenie firmy
% - [ ] Ile powinniśmy mieć aby firma mogła regulować wydatki na bieżąco?
% - [ ] Jak uzyskać fundusze?

% - [ ] Elementy planu finansowego
%   - [ ] bilans
%   - [ ] założenia
%   - [ ] rachunek zysków i strat
%   - [ ] rachunek przepływów pieniężnych

% - [ ] Zadaj sobie kilka pytań:
%   - [ ] Ile pieniędzy firma potrzebuje i w jakim okresie?
%   - [ ] Jakich zysków można się spodziewać po wprowadzeniu firmy na rynek?
%   - [ ] Na jakich założeniach oparte są prognozowane wyniki?

\end{document}
