\documentclass[12pt]{article}

\usepackage[utf8]{inputenc}
\usepackage[T1]{fontenc}
\usepackage[polish]{babel}
\usepackage{polski}
\selectlanguage{polish}

\usepackage[backend=biber]{biblatex}
\addbibresource{bibliografia.bib}
\usepackage{vwcol}
\usepackage[letterspace=20]{letterspace}
\usepackage{booktabs}
\usepackage{multirow}
\usepackage{makecell}
\usepackage[gen]{eurosym}
\usepackage{amstext}
\usepackage{xspace}
\usepackage{csquotes}

\newcommand{\tabitem}{~~\llap{\textbullet}~~}

\usepackage{hyperref}

\DeclareUnicodeCharacter{00A0}{~}

\newcommand{\productpricezl}{299 zł\xspace}
\newcommand{\nazwafirmy}{aiDrums\xspace}

\begin{document}

\begin{titlepage}
    \hspace{0em}
    \vfill
    \begin{vwcol}[widths={0.2,0.8}, sep=2em, rule=0em]
        {\normalsize \textbf{\uppercase{Biznesplan}}}

        \vfill\eject

        {\LARGE \lsstyle{Wirtualna perkusja Mike'a Malyana}} \newline \vspace{1em}
        {\large Tomasz Witczak, 216920} \newline \vspace{1em}
        {\normalsize \today}
    \end{vwcol}
    \vfill
    \hspace{0em}
\end{titlepage}

\tableofcontents

\listoftables

\newpage

\section{Streszczenie}

% - [ ] Streszczenie jest zwięzłą, miniaturową wersją całego biznesplanu i stanowi zamkniętą całość
% - [ ] Nie należy go utożsamiać ze wstępem ani wprowadzeniem
% - [ ] Jest to mały biznesplan wewnątrz większego
% - [ ] Streszczenie należy napisać z wielką starannością, ponieważ jego niska jakość może skutecznie zniechęcić potencjalnego inwestora
% - [ ] Należy używać zrozumiałego języka i precyzyjnie wyjaśnić dlaczego właśnie to przedsięwzięcie powinno zostać sfinansowane

% - [ ] Objętość streszczenia to ok. 1-4 strony
% - [ ] Zawiera ono najważniejsze kwestie, kluczowe dane i argumenty mające poinformować czytającego o przedsięwzięciu i przekonać go, że ta firma odniesie sukces
% - [ ] Streszczenie choć umieszczone na początku, jest sporządzane już po przygotowaniu biznesplanu

% - [ ] Streszczenie należy zacząć od zwięzłego opisu pomysłu, który ma zainteresować potencjalnych inwestorów
% - [ ] Powinno zawierać
%   - [ ] nazwę przedsiębiorstwa
%   - [ ] nazwiska właścicieli
%   - [ ] koncepcję rozwoju firmy
%   - [ ] przewidywane koszty i zyski
%   - [ ] plan wykorzystania środków inwestycyjnych

% - [ ] W streszczeniu powinno się znaleźć
%   - [ ] wyjaśnij co firma oferuje, na jakim rynku, z kim konkuruje
%   - [ ] wskaż dlaczego osiągniesz sukces, jakie czynniki o tym zadecydują
%   - [ ] określ jak wysokie prognozujesz zyski, jaki będzie kierunek rozwoju firmy

% - [ ] W streszczeniu biznesplanu powinny znaleźć się tylko to co istotne. Niezbędne minimum to:
%   - [ ] firma
%   - [ ] zespół
%   - [ ] oferta
%   - [ ] czynniki decydujące o sukcesie przedsięwzięcia
%   - [ ] przewidywane zyski
%   - [ ] główne ryzyko i sposoby jego minimalizacji
%   - [ ] oczekiwania w stosunku do czytającego

% - [ ] W streszczeniu trzeba opisać
%   - [ ] siebie
%   - [ ] swój zespół
%   - [ ] historię
%   - [ ] czym się zajmujesz
%   - [ ] co jest takiego w Twojej ofercie
%   - [ ] że odniesiesz sukces
%   - [ ] jaki osiągniesz zysk
%   - [ ] czego oczekujesz od czytającego

% ! [x] Musisz jednak pisać zwięźle i konkretnie
% ! [x] Uważa się, że czytelnik powinien przeczytać i zrozumieć streszczenie w ciągu 5-10 minut
% - [ ] Można to sprawdzić dając je do przeczytania komuś kto nic nie wie o przygotowanej koncepcji i poprosić o krótkie zreferowanie tego co się dowiedział

% - [ ] Streszczenie powinno zawierać odpowiedzi na 5 pytań
%   - [ ] Czy istnieje rynek na danego typu produkty lub usługi?
%   - [ ] Czy przedsiębiorstwo oferuje produkty lub usługi pozwalające zdobyć rynek?
%   - [ ] Czy przedsiębiorstwo dysponuje ludźmi zdolnymi poprowadzić je do sukcesu?
%   - [ ] Jakie zyski przedsiębiorstwo będzie przynosić?
%   - [ ] W jakim kierunku przedsiębiorstwo będzie zmierzać?

% ! [x] Powinno być na tyle interesujące dla czytelników, aby chcieli przeczytać resztę dokumentu
% ! [x] Można je rozpocząć od jakiegoś poruszającego wyobraźnię stwierdzenia, które ukazuje zaobserwowaną okazję
% - [ ] Przygotowując jakąś usługę dla osób starszych można zacząć na przykład w ten sposób: Główny Urząd Statystyczny przewiduje, że w ciągu najbliższych 20 lat liczba ludzi w wieku powyżej 80 lat wzrośnie dwukrotnie
% ! [x] Czytający odniesie wrażenie, że jest potencjał do wykorzystania

\paragraph{Kiedy? | Wstęp}

Rozwój technologii zwiększa dostęp do wysoce jakościowych metod nagrywania i produkcji utworów dla coraz większej liczby osób w branży muzycznej.
Miniaturyzacja i spadek cen interfejsów audio oraz programów do edycji dźwięku wpływają na rosnącą liczbę powstających studiów domowych.
Profesjonaliści oraz amatorzy mogą tworzyć i rejestrować piosenki korzystając ze swoich komputerów.

Ważnym elementem muzyki popularnej jest perkusja.
Zwykle, aspirujący kompozytor musiał udać się do studia muzycznego z zatrudnionym perkusistą w celu nagrania partii.
Obecnie, popularność zyskały tzw. wirtualne instrumenty perkusyjne -- biblioteki sampli.
Producent programuje linię rytmiczną, zaznaczając kiedy i z jaką artykulacją powinien wybrzmieć dany element zestawu, a odegraniem muzyki zajmuje się komputer.

% ! [x] Możesz napisać np. o tym
%   ! [x] jak nazywa się firma
%   - [ ] w jaki sposób będzie zarabiać
%   - [ ] jaka jest jej misja
%   - [ ] kim są klienci docelowi
%   - [ ] co im oferuje i czym to się różni od konkurencji
%   - [ ] w jaki sposób zamierzasz sprostać konkurencji
%   - [ ] jak duży jest rynek
%   - [ ] ilu klientów zamierzasz pozyskać w ciągu pierwszych lat
%   - [ ] jaką przyjmiesz strategię sprzedaży i dystrybucji
%   - [ ] jaki masz slogan reklamowy
%   - [ ] jak wygląda struktura przychodów
%   - [ ] jaki jest plan promocji i działań marketingowych
%   - [ ] jakie są cele sprzedażowe firmy i jak je osiągniesz
%   - [ ] jak wyglądają prognozy sprzedaży
%   - [ ] kiedy osiągniesz próg rentowności
%   - [ ] jakie masz źródła finansowania
%   - [ ] jakie będą potrzeby finansowe w najbliższych latach
%   - [ ] z kim będziesz współpracować

\paragraph{Kto i co? | Firma i oferowany produkt}

Firma \textit{\nazwafirmy} wychodzi naprzeciw oczekiwaniom konsumentów i oferuje innowacyjne wydanie sprawdzonego schematu produktu.

Przedmiotem inwestycji jest wirtualny instrument perkusyjny sygnowany nazwiskiem brytyjskiego muzyka Mike'a Malyana.
Opracowane oprogramowanie pozwoli na symulację brzmienia zestawu artysty.
Za realizm odpowiedzialna będzie sztuczna inteligencja trenowana specjalnie w celu otworzenia niuansów gry oraz bogactwa artykulacji.
Będzie to pierwszy produkt z tego segmentu wykorzystujący sieci splotowe do analizy i syntezy audio.
Poprzez atrakcyjny i nowoczesny, jednakże prosty i znajomy interfejs użytkownik zyska dostęp do nagrań gotowych do użycia w profesjonalnych produkcjach muzycznych.

\paragraph{W jaki sposób? | Produkcja}

Cała inwestycja rozpoczyna się od wkładu wspólników firmy.
Po przeprowadzeniu procesu nagrań i ich edycji, zespół programistów przystąpi do implementacji i testowania.
Nad wszystkimi pracownikami będą czuwać dyrektorzy działów, m.in. ds. marketingu i sprzedaży internetowej czy też ds. produkcji muzycznej

Kadra będzie składać się wyłącznie ze specjalistów w swojej dziedzinie.
Zagwarantuje to nam wysoką jakość końcową produktu oraz sprawny przebieg procesów wewnątrz firmy.

\paragraph{Dla kogo? | Rynek i grupa docelowa}

Grupą docelową są młodzi muzycy i producenci, z których składa się nisza branży muzycznej -- \emph{homerecording}.
Mimo istnienia wielu firm oferujących podobne produkty, uważamy, że branży daleko do nasycenia i jest w niej ciągle miejsce na innowacje.

Pasjonaci dobrego brzmienia, profesjonaliści i amatorzy, głównie należący do klasy średniej.
Sądzimy, że takie osoby zadowolą się naszym produktem i wyrażą o nim pozytywne opinie.

\paragraph{Którędy? | Marketing internetowy}

W celach marketingowych wykorzystamy nowoczesne media.
Zbudujemy obecność marki w Internecie -- na profilach i grupach w mediach społecznościowych oraz mailowo poprzez newsletter.
Zadbamy o pozycjonowanie, reklamy oraz artykuły.
Stworzymy pomocne materiały udostępniane w formie filmów instruktażowych.

\paragraph{Jak? | Sprzedaż}

Sprzedaż będzie odbywać się wyłącznie poprzez globalną stronę internetową produktu.
Wdrożony system e-commerce pozwoli na bezpieczne transakcje i możliwość skorzystania z różnych metod zapłaty.
Po uiszczeniu opłaty, klient otrzyma odnośniki plików oprogramowania do pobrania.

\paragraph{Ile? | Zysk}

Po analizie popularności firm konkurencyjnych, wprowadzimy nasz produkt w dostosowanej do warunków cenie \productpricezl.
Zakładamy sukces kampanii marketingowej oraz dużą popularność produktu w pierwszym roku.
Liczymy na ok. 25 tys. sprzedanych egzemplarzy co przeniesie się na ok. 3 mln zł zysku przez okres 4 lat działalności.

\paragraph{Co dalej? | Utrzymanie klientów i konserwacja produktu}

W celu utrzymania klientów zaproponujemy ciekawą zawartość udostępnianą poprzez kanały społecznościowe.
Zostanie uruchomiony program poleceń oferujący zniżki dla nowych klientów.

Jesteśmy otwarci na wszystkie opinie i ich publiczne wyrażanie.
Klienci będą mieć do wyboru kontakt poprzez media społecznościowe oraz firmowy adres mailowy.

W przypadku wystąpienia problemów zapewniamy dostęp do wsparcia technicznego -- zarówno odnośnie procesu kupna jak i funkcjonowania aplikacji.
Oprogramowanie będzie regularnie aktualizowane przez specjalistów.

Jeśli wirtualny instrument odniesie sukces finansowy, zarząd rozważy wypuszczenie na rynek kolejnych produktów.

\section{Opis produktu}

% - [ ] Firmę prowadzi się po to, aby zarabiać na sprzedaży produktów lub usług
% ? [ ] Ich opis jest podstawowym celem biznesplanu
% - [ ] Jego czytelnik chce wiedzieć co planujesz sprzedawać i dlaczego klienci będą chcieli to kupić
% ? [ ] Należy zastanowić się jakie potrzeby produkt zaspokaja

% - [ ] Ważny jest innowacyjny produkt lub usługa, którą chcesz wprowadzić na rynek i określenie czym będą się one różniły, a raczej wyróżniały na tle tych oferowanych przez konkurencję
% ? [ ] Opisz produkt podając jego rozmiary, kolory, wysokość, wagę funkcje
% - [ ] Należy omówić funkcje jakie spełnia produkt i korzyści uzyskane przez klienta dzięki niemu

% ? [ ] Odpowiedz sobie na kilka pytań:
%   ? [ ] Kto jest odbiorcą?
%   - [ ] Co oferuję klientowi?
%   ? [ ] Jacy kooperanci są mi potrzebni?
%   - [ ] Jakie konkurencyjne produkty są już na rynku?
%   ? [ ] Na czym polega innowacyjność mojej oferty?

% ? [ ] Zacznij od ogólnej charakterystyki oferty i potem przejdź do szczegółów
% ? [ ] Opisz swój produkt lub usługę, a także branżę w jakiej działasz, jakie są koszty produktu
% ? [ ] Napisz w jaki sposób go pozyskasz
% - [ ] Jeśli zamierzasz sam go wyprodukować wskaż komponenty, z których jest wytwarzany i określ koszty
% - [ ] Jeśli jest to usługa napisz kto wykonuje konkretne prace i gdzie

% - [ ] Jeśli to ułatwi zrozumienie możesz zawrzeć ilustracje sprzedawanego towaru
% - [ ] Warto też opisać doświadczenie jakim dysponujesz, a które legitymizuje Twoją zdolność do sprzedaży
% - [ ] Wskaż kierunki rozwoju działalności, które są możliwe dla tego produktu/usługi

% ? [ ] Produkt powinien zostać opisany możliwie najdokładniej, ponieważ osoby z zewnątrz nie znają jego specyfiki
% - [ ] Dokonując charakterystyki należy uwzględnić następujące informacje:
%   - [ ] rodzaj produktu,
%   - [ ] jego główne cechy i zalety,
%   - [ ] wyjaśnienie czy jest to produkt, który już istnieje, czy jest dopiero projektowany,
%   - [ ] pokazanie czym produkt różni się od innych dostępnych na rynku, na czym polega jego wyjątkowość,
%   - [ ] koszt jednostkowy produktu,
%   - [ ] cena produktu,
%   - [ ] zysk jednostkowy,
%   - [ ] posiadane znaki towarowe,
%   - [ ] sposób zapewnienia jakości produktu,
%   - [ ] wskazanie czy wyrób stwarza możliwość jego modyfikacji czy ulepszeń przy stosunkowo niewielkich zmianach,
%   - [ ] planowana wielkość produkcji (miesięczna, kwartalna, roczna)

% ? [ ] Jeśli masz takie dane, napisz również co myślą o produkcie klienci

% ? [ ] Co powinno się znaleźć w opisie produktu/usługi
%   ? [ ] fizyczny opis produktu (m.in. kolor, rozmiar, waga, funkcje, wykonanie)
%   ! [x] zalety produktu, dlaczego klient będzie chciał go kupić, co go wyróżnia na tle innych
%   ? [ ] cena, koszt pozyskania, wielkość produkcji, metody zapewnienia wysokiej jakości

\subsection{Produkt}

% ? <Krótki, ogólny opis produktu>
% ? <Kto jest odbiorcą?>

Inwestycja dotyczy biblioteki sampli perkusyjnych sygnowanych przez muzyka Mike'a Malyana.
Produkt nie jest pierwszym takim na rynku, jednakże bazując na sukcesach i porażkach konkurencji, rozwija pomysł i wprowadza nową funkcjonalność.
Dostarczony w formie wtyczek VSTi, zadziała na każdym systemie operacyjnym i w każdym głównym programie DAW.
Użycie sztucznej inteligencji do humanizacji odsłuchu zapewni klientowi jakość, której próżno szukać u konkurencji.

\subsection{Charakterystyka produktu}

% ? <Szczegółowy opis produktu od strony wizualnej/dźwiękowej/użytkowej>
% ? <Opis branży>

Biblioteka będzie zawierać nagrania poszczególnych instrumentów, z których składał się zestaw perkusyjny użyty przez Mike'a Malyana w 2017 roku na płycie ,,Foreword'' zespołu Disperse.
Kolekcja skierowana jest na prostotę użycia, stąd znajdą się w niej tylko niezbędne elementy takie jak: werbel, stopa, trzy tomy i talerze.
Każdy instrument zawierać będzie nagrania wykonane wielokrotnie w różnych konfiguracjach.
Uzyskane dzięki temu sample pozwolą na dobre odwzorowanie dynamiki i artykulacji.

Interfejs do biblioteki zostanie udostępniony w formie programu VST.
Taką wtyczkę będzie można bez problemu zaimportować w większości używanych dzisiaj stacji muzycznych.
Użytkownikowi zostanie przedstawiona uproszczona wizualizacja perkusji z możliwością dostosowywania opcji za pomocą suwaków.
Dołączone zostaną także gotowe mapowania MIDI umożliwiające dostosowanie używanych artykulacji pod preferencje klienta.
Po załadowaniu instrumentu na ścieżkę, należy dołączyć plik MIDI z sekwencją nut w celu odegrania dźwięków.

\subsection{Innowacyjność produktu}

% ? <Jakie potrzeby zaspokaja produkt?>
% ? <Dlaczego klient będzie chciał kupić produkt?>
% ? <Innowacyjność>

Produkt jest jedynym na rynku, który uwzględnia pełen zestaw perkusyjny Mike'a Malyana.

Istotnie, nowatorskie użycie sieci splotowych jako sztucznej inteligencji zapewnia niespotkaną wcześniej humanizację dźwięków.
Algorytm został nauczony na bazie prawdziwych nagrań Malyana i dzięki temu zapewnia odwzorowanie niuansów jego gry w programie.

Dzięki naszemu produktowi możesz mieć realnie brzmiące ścieżki perkusyjne na swoim albumie przy niskich kosztach.

\subsection{Zalety i wady}

\paragraph{Zalety}

\begin{itemize}
    \item pełen zestaw perkusyjny Malyana
    \item idealne odwzorowanie niuansów i detali (dynamika, timing i artykulacja)
    \item wysoka jakość brzmienia
    \item gotowość ścieżek do miksu/masteringu (minimalna wymagana obróbka)
    \item sztuczna inteligencja
    \item niskie zużycie pamięci RAM
    \item atrakcyjny i nowoczesny, aczkolwiek prosty interfejs użytkownika
\end{itemize}

\paragraph{Wady}

\begin{itemize}
    \item ukierunkowanie na niszę konsumencką -- idealny klient to muzyk szukający nowoczesnego brzmienia perkusji
    \item ubogość zestawu -- jeśli klientowi zależy na dużym wyborze brzmień, ten zestaw nie jest dla niego
\end{itemize}

\subsection{Koszty i komponenty}

% ? <Z kim będę współpracował? W jaki sposób pozyskam produkt?>
% ? <Koszty produktu (pozyskania), wielkość produkcji, metody zapewnienia wysokiej jakości>

Produkt będzie wymagał początkowej odgórnej inwestycji finansowej.
Produkcja odbędzie się raz przed rozpoczęciem sprzedaży.
Dystrybucja dotyczyć będzie cyfrowych kopii oprogramowania.

Na wstępne koszty (tablica \ref{table:jednorazowe}) składać się będzie m.in.:

\begin{itemize}
    \item podpisanie umowy z muzykiem
    \item zatrudnienie inżyniera oraz wynajęcie studia w Anglii
    \item proces nagrania i przygotowania sampli
    \item wytworzenie oprogramowania i nauka sztucznej inteligencji
    % \item marketing internetowy
    % \item wsparcie dla produktu w przypadku wystąpienia błędów (aktualizacje oprogramowania)
\end{itemize}

\section{Zespół zarządzający}

% ! [x] Należy zawrzeć:
%   ! [x] charakterystyki osób zarządzających firmą,
%   ! [x] ich doświadczenie,
%   ! [x] wykształcenie i osiągnięcia zawodowe
% ! [x] Trzeba omówić te kwalifikacje, które są niezbędne dla realizacji biznesplanu
% - [ ] Warto też wspomnieć o doradcach, takich jak: księgowi, firmy PR, itp.

\subsection{Struktura kierownicza firmy}

Wstępny kapitał firmy pochodzi z wkładu udziałowców (tablica \ref{table:wspolnicy}).
Za sukces produktu będzie odpowiadać kadra dyrektorska składająca się ze specjalistów (tablica \ref{table:struktura-organizacyjna-zadania-kierownicze}).

Dyrektorem naczelnym oraz głównym udziałowcem jest Tomasz Witczak.
Wykształcony specjalista z branży informatycznej oraz pasjonat muzyki.
W wolnych chwilach producent i kompozytor, profesjonalnie -- analityk danych i znawca sztucznej inteligencji.

Łukasz Ziułkiewicz to dyrektor ds. produkcji muzycznej.
Absolwent renomowanej prywatnej szkoły realizatorskiej, od wielu lat pracuje w branży nagraniowej.

Marketing i sprzedaż internetową nadzorować będzie Jan Kowalski, kolejny specjalista z branży IT.
Doświadczony oraz lubiany przez współpracowników.

Za stronę techniczną będzie odpowiadać Michał Rogala.
W trakcie swojej pracy zawodowej miał do czynienia z podobnymi projektami, a jego doświadczenie jest nieocenione.

Kontrolą jakości zajmie się dyrektor Albert Pietrzak.
Perkusista i znawca muzyczny, w swojej ciągle aktywnej karierze wydał już wiele płyt.
Zna się na brzmieniach perkusyjnych i technologiach realizatorskich.

\begin{table}[h!]
    \begin{center}
        \makebox[\linewidth]{%
            \begin{tabular}{llcr}
                \toprule
                \multicolumn{1}{c}{\textbf{Imię}} & \multicolumn{1}{c}{\textbf{Nazwisko}} & \multicolumn{1}{c}{\textbf{Rodzaj wkładu}} & \multicolumn{1}{c}{\textbf{Wartość}} \\
                \midrule
                Tomasz & Witczak     & pieniądze & 150 tys. zł \\
                Łukasz & Ziułkiewicz & pieniądze & 50 tys. zł  \\
                Mike   & Malyan      & perkusja  & 40 tys. zł  \\
                \bottomrule
            \end{tabular}}
            \caption{Wspólnicy i wnoszone wkłady}
            \label{table:wspolnicy}
    \end{center}
\end{table}

\begin{table}[h!]
    \begin{center}
        \makebox[\linewidth]{%
            \begin{tabular}{p{5cm}llp{7cm}}
                \toprule
                \multicolumn{1}{c}{\textbf{Funkcja}} &
                    \multicolumn{1}{c}{\textbf{Imię}} &
                    \multicolumn{1}{c}{\textbf{Nazwisko}} &
                    \multicolumn{1}{c}{\textbf{Zadania}} \\
                \midrule
                dyrektor naczelny                                & Tomasz & Witczak     & nadzór nad przebiegiem pełnego procesu od projektu, poprzez wykonanie, wdrożenie i wsparcie produktu \\
                \midrule
                dyrektor ds. marketingu i sprzedaży internetowej & Jan    & Kowalski    & odpowiada za wypromowanie marki w mediach społecznościowych oraz przyciągnięcie klientów             \\
                \midrule
                dyrektor ds. inżynierii oprogramowania           & Michał & Rogala      & zajmuje się nadzorem kwestii związanych z technologicznym aspektem wtyczki VSTi                      \\
                \midrule
                dyrektor ds. produkcji muzycznej                 & Łukasz & Ziułkiewicz & jako producent muzyczny, zarządza początkowym etapem nagrań perkusyjnych oraz ich obróbką            \\
                \midrule
                dyrektor ds. kontroli jakości                    & Albert & Pietrzak    & jako doświadczony muzyk, przeprowadza proces kontroli jakości oprogramowania pod kątem muzycznym     \\
                \bottomrule
            \end{tabular}}
            \caption{Struktura organizacyjna i zadania kierownicze}
            \label{table:struktura-organizacyjna-zadania-kierownicze}
    \end{center}
\end{table}

\section{Rynek i konkurencja}

\subsection{Rynek}

\subsubsection{Wielkość rynku i możliwości jego rozwoju}

% ! [x] Wielkość rynku powinna być podana w liczbach
%   ! [x] liczby klientów
%   ? [ ] liczby sprzedanych jednostek
%   ! [x] spodziewanych obrotów
% ! [x] Uwzględniamy przewidywania co do rozwoju rynku, wskazujemy czynniki istotne dla tego rozwoju
% ? [ ] W tym celu można sięgnąć do literatury fachowej, baz teleadresowych, badań rynku, stowarzyszeń czy zrzeszeń, Internetu, wywiadów, itp.

Wraz z rozwojem technologii, obecnie coraz więcej osób w branży muzycznej zyskuje dostęp do wyrafinowanych metod nagrywania i produkcji utworów.
Dzieje się tak za sprawą miniaturyzacji i spadku cen m.in. interfejsów audio, które poprzez podłączenie do komputera pozwalają na cyfrowe nagrywanie takich instrumentów jak np. gitary elektryczne czy keyboardy.
Pozwala to często na przeniesienie znaczącej części realizacji nagrań -- a nawet w pełni -- do tzw. studiów domowych.

Za pomocą programów do obróbki dźwięku, tzw. DAW, osoby niebędące profesjonalnymi producentami muzycznymi mogą zająć się tworzeniem własnych piosenek.
Do tego celu często wykorzystuje się m.in. \emph{programowanie perkusji}.
Autor może rozpisać ścieżkę dźwiękową przy pomocy MIDI, w której zawiera informację o tym, kiedy dana część zestawu powinna zostać odegrana, z jaką głośnością i zakresem dynamiki.
Kwestią odtworzenia odpowiednich dźwięków i jednocześnie symulacji grania na prawdziwej perkusji zajmuje się komputer.

% ? <Wielkość rynku>
% ? <Co myślą o podobnych produktach klienci?>
Przeniesienie dźwięków zestawu perkusyjnego do domeny cyfrowej jest coraz bardziej powszechne i popularne wśród profesjonalistów i amatorów.
Powstaje coraz większa liczba firm oferujących sample (próbki, nagrania) perkusyjne pasujące do różnych gatunków muzyki.
Tysiące klientów na całym świecie chwali sobie bogactwo dostępnych od ręki brzmień, prostotę użytkowania produktu oraz przystępne ceny.

% ? <Przewidywania rozwoju rynku w przyszłości>
Uważamy, że ten rejon branży muzycznej rozwija się w znaczącym tempie, a inwestycja może przynieść duży zysk.
Sądzimy, że zapotrzebowanie na realistyczne brzmienia perkusyjne będzie rosło w najbliższych latach.
Pandemia wirusa SARS-CoV-2 zmusza ludzi do pozostania w domach, stąd popyt na technologie zapewniające muzykom wygodę komponowania i wysoką jakość nagrań.

% ? <Odnośniki do danych>

\subsubsection{Segmentacja rynku}

% ! [x] Liczbę klientów i ich zachowanie w konkretnym segmencie rynku należy zdefiniować i udokumentować
% ! [x] Celem segmentacji jest wybór grupy docelowej dla produktu
% ! [x] Należy określić przewidywaną wielkość sprzedaży
% ! [x] Segmentacji można dokonać na podstawie
%   ! [x] miejsca zamieszkania
%   ! [x] danych demograficznych
%   ! [x] typowych zachowań
%   ! [x] zamożności

% ? <Grupa docelowa i jej cechy>
% ? <Liczba klientów i jej zachowanie w segmencie>
Grupą docelową naszego produktu będą głównie \emph{młodzi muzycy i producenci}:
\begin{itemize}
    \item szukający sampli wysokiej jakości
    \item skoncentrowani na komponowaniu i nagrywaniu w domu
    \item będący fanami twórczości Mike'a Malyana i brzmienia jego zestawu perkusyjnego
    \item mieszkający w miejscu z dostępem do Internetu
    \item należący do klasy średniej
\end{itemize}

% ? <Przewidywana wielkość sprzedaży>
Ograniczając się tylko do szacowanej liczby osób w grupie docelowej oraz wnioskując z popularności produktów konkurencji, uważamy, że produkt naszej firmy ma szansę na sprzedaż w liczbie ok. \emph{25 tys. egzemplarzy}.

\subsection{Konkurencja}

% ! [x] Określić głównych konkurentów i zbadać ich pozycję na rynku
% ! [x] Kryteria
%   ! [x] wysokość obrotów
%   ! [x] wielkość przychodów ze sprzedaży
%   ! [x] wysokość cen
%   ! [x] udział w rynku
%   ! [x] kanały dystrybucji
%   ! [x] itp.

% ? <Jacy są główni konkurenci i ich pozycja na rynku?>
Na rynku sampli perkusyjnych istnieje obecnie kilka rozpoznawalnych firm, które zestawiliśmy w tablicy \ref{table:competition}.

Z braku oficjalnych danych, za miarę popularności firmy i wysokości obrotów przyjęliśmy \emph{statystyki polubień w serwisach społecznościowych\footnote{Liczba polubień na Facebooku\label{footnote:facebook}}\textsuperscript{,}\footnote{Liczba obserwujących na Instagramie\label{footnote:instagram}}}.
Media tego typu są głównym kanałem marketingowym firm w branży muzycznej.
Dystrybucja produktów odbywa się obecnie \emph{wyłącznie internetowo oraz cyfrowo}.
Odnośniki do kopii plików nagrań oraz odpowiednich programów przesyłane są do klienta drogą mailową po uiszczeniu opłaty.

Na bazie danych z mediów społecznościowych, średniej ceny za oferowany produkt u danego producenta oraz średniego kursu waluty (na dzień 2 listopada 2020) oszacowaliśmy możliwe zyski danej firmy\footnote{$\text{szacowany zysk} = \text{liczba polubień} \cdot \text{średnia cena produktu} \cdot \text{kurs waluty}$ \\ Dane na dzień 2 listopada 2020.\label{footnote:income}}.
Nie jest to dokładna metoda, lecz pozwala na przewidzenie możliwych obrotów naszego produktu.
Strony społecznościowe dużych firm, jak \textit{Native Instruments}, są obserwowane przez wiele osób, które niekoniecznie zakupiły ich zestaw perkusyjny.
Uważamy, że najbardziej wiarygodny wynik szacunków uzyskaliśmy dla \textit{GetGood Drums}.
Na dzień dzisiejszy ta firma sprzedaje głównie zestawy sampli perkusyjnych.
Przyjmując, że każdy obserwujący zakupił u nich co najmniej jeden zestaw, szacujemy ich zysk na minimum 10 mln zł.

Model biznesowy i podejście do tematu różni się w zależności od producenta.
Dla większych firm, takich jak \textit{Native Instruments}, wirtualne perkusje są tylko jednym z oferowanych produktów.
Dla tych mniej popularnych, może być to główne źródło utrzymania.
Niektóre firmy, jak \textit{XLN Audio} i \textit{Toontrack}, stawiają na droższe, pojedyncze, duże kolekcje brzmień, natomiast inne -- \textit{GetGood Drums} -- tworzą kilka osobnych, tańszych produktów, z których każdy zawiera zwykle jeden pełen zestaw perkusyjny.

Do stworzenia zestawu sampli niekoniecznie potrzebna jest duża firma z zespołem wielu profesjonalistów.
Coraz częściej pojedynczy producenci muzyczni i właściciele studiów, np. \textit{Taylor Larson}, sprzedają wysokiej jakości sample, które zwykle są nagrywane np. przy okazji realizacji płyt dla klientów.

Analizując sukces i rosnącą popularność produktów firmy \textit{GetGood Drums}, uważamy, że kluczem do kreacji dobrego produktu jest głównie zapewnienie wysokiej jakości brzmienia i obszerna oferta realizowana poprzez wiele niezależnych produktów. Sądzimy, że tańsze, aczkolwiek wyspecjalizowane zestawy trafiają lepiej w gusta naszej grupy docelowej.

\begin{table}[h!]
 \begin{center}
 \makebox[\linewidth]{%
 \begin{tabular}{lrlrr}
 \toprule
 \textbf{Producent} & \textbf{Polubienia} & \textbf{Produkty} & \textbf{Cena} & \textbf{Zysk}\textsuperscript{\ref{footnote:income}} \\
 \midrule
 Native Instruments & 387 852\textsuperscript{\ref{footnote:facebook}} & Studio Drummer & \euro 149 & 266 mln zł \\
 \midrule
 Slate Digital & 187 937\textsuperscript{\ref{footnote:facebook}} & Steven Slate Drums 5.5 & \$149 & 111 mln zł \\
 \midrule
 Toontrack & 75 267\textsuperscript{\ref{footnote:facebook}} & Superior Drummer 3 & \euro 379 & 131 mln zł \\
 \midrule
 \multirow{5}{*}{GetGood Drums} & \multirow{5}{*}{28 446\textsuperscript{\ref{footnote:facebook}}} & One Kit Wonder: Modern Fusion & \$59 & \multirow{5}{*}{10 mln zł} \\
 & & Matt Halpern Signature Pack & \$89 & \\
 & & Modern \& Massive Pack & \$99 & \\
 & & P IV Matt Halpern Signature Pack & \$99 & \\
 & & Invasion & \$119 & \\
 \midrule
 Taylor Larson & 22 799\textsuperscript{\ref{footnote:instagram}} & TJL 2.0 Drum Sample Pack & \$69 & 6 mln zł \\
 \midrule
 \multirow{2}{*}{XLN Audio} & \multirow{2}{*}{19 311\textsuperscript{\ref{footnote:facebook}}} & Addictive Drums 2 Basic & \euro 169 & \multirow{2}{*}{39 mln zł} \\
 & & Addictive Drums 2 Complete Collection & \euro 699 & \\
 \bottomrule
 \end{tabular}}
 \caption{Zestawienie najbardziej znanych produktów firm konkurencyjnych (stan na 02.11.2020)}
 \label{table:competition}
 \end{center}
\end{table}

% ? <Oszacuj wysokość obrotów i przychodów, podaj ceny, udział w rynku, dystrybucję>

\section{Plan marketingowy}

% ! [x] Plan marketingowy określa zestaw posunięć firmy w strefie marketingu i pozwala wyrobić sobie pogląd na temat przyjętej strategii
% ! [x] Określa on co, komu, kiedy i jak firma sprzedaje
% ! [x] Jego uzupełnieniem jest plan sprzedaży, który z kolei opisuje w jaki sposób i komu są sprzedawane produkty lub usługi.

% ! [x] Działania marketingowe i sprzedaż produktów są jednymi z najistotniejszych elementów powodzenia całego przedsięwzięcia
% ! [x] Najważniejsze tutaj to wziąć pod uwagę kilka czynników takich jak
%   ! [x] produkt
%   ! [x] cena
%   ! [x] miejsce i sposób dystrybucji
%   ! [x] promocja i reklama
% ! [x] Są one potocznie nazywane mieszanką marketingową. W planie marketingowym należy je scharakteryzować
% - [ ] Często wyznaczają jego strukturę, stanowią kolejno omawiane podpunkty

% - [ ] Przy planowaniu działań trzeba uwzględnić budżet
% - [ ] Dobrze skalkulować na co firmę stać a na co nie
% - [ ] W przypadku niewielkich środków najlepiej skierować przekaz bezpośrednio do grupy docelowej

% ! [x] Właściwie napisany plan określa z jakich narzędzi będziesz korzystać, aby dotrzeć do klientów
% ! [x] Powinno się
%   ! [x] dokładnie wskazać metody reklamy i promocji używane do sprzedaży
%   ! [x] napisać kiedy, gdzie, dlaczego, w jaki sposób zamierza się dotrzeć do klientów i jakie to pociągnie za sobą koszty

% - [ ] Istnieje wiele sposobów reklamy i promocji
% - [ ] Można skorzystać z gazet, broszur, katalogów, telewizji i radia, ulotek reklamowych, informacji prasowych, tablic ogłoszeń
% ! [x] W Internecie wykupić usługę pozycjonowania w wyszukiwarkach, rozsyłać promocyjne e-maile, biuletyny
% ! [x] Wszystkie te narzędzia mogą zostać użyte wspólnie w ramach 4 podstawowych platform marketingowych
%   ! [x] pozycjonowania
%   ! [x] pozyskiwania klientów
%   ! [x] ich utrzymania
%   ! [x] strategii osiągania przychodów

\subsection{Obecność firmy w Internecie} % \subsection{Marketing internetowy}

% ! [x] We współczesnym świecie bardzo ważna jest obecność firmy w Internecie
% ! [x] Należy zadbać o to, aby było ją łatwo znaleźć
% ! [x] Należy zastanowić się w jaki sposób dotrzeć do szerokiej rzeszy klientów, którzy korzystają z tego medium

W obecnych czasach, obecność firmy w Internecie jest kluczowa.

Biorąc pod uwagę tendencje branży muzycznej w segmencie wirtualnych instrumentów, Internet będzie wyłącznym medium marketingowym dla naszego produktu.
Istotnie, po utworzeniu stron w Internecie oraz w mediach społecznościowych, musimy zadbać o odpowiednie pozycjonowanie w wyszukiwarkach.
Głównym źródłem pozyskiwania i utrzymywania klientów są media społecznościowe, więc na ich prowadzenie przeznaczymy regularne środki.

Koszty poszczególnych etapów i działań marketingowych przedstawiliśmy w tablicy \ref{table:marketing-sprzedaz}.

% ! [x] Do pozycjonowania można zatrudnić profesjonalną firmę zewnętrzną
% ! [x] Jeśli nie ma się doświadczenia jest to dobre wyjście, ponieważ nieumiejętne pozycjonowanie może bardziej zaszkodzić niż pomóc

\subsubsection{Zespół ds.\ marketingu i sprzedaży internetowej}

Kwestiami marketingu internetowego zajmie się nasz dyrektor do spraw marketingu i sprzedaży internetowej -- Jan Kowalski -- wraz ze swoim kilkuosobowym zespołem.
Mimo obecności na rynku wielu firm zewnętrznych, zdecydowaliśmy się na zlecenie tego zadania osobie z wewnątrz naszej firmy.
W ten sposób nastąpi redukcja potencjalnych kosztów a utrzymywanie stron oraz dostosowywanie się do warunków rynku będzie sprawniejsze poprzez wewnętrzny iteracyjny charakter pracy.

Jesteśmy świadomi, że nieumiejętne podejście do tego zagadnienia może bardziej zaszkodzić niż pomóc, dlatego zaufaliśmy profesjonaliście.
Pan Kowalski od kilkunastu lat pracuje w branży IT a jego doświadczenie będzie w tej sprawie nieocenione.
Przez kilka lat pan Kowalski zajmował się tworzeniem stron internetowych oraz świadczył usługi SEO (pozycjonowanie).

% ! [x] Dopilnuj, aby na materiałach firmowych był zamieszczony adres strony internetowej i adres e-mailowy
% ! [x] Napisz o tym, że np. zapewnisz klientom możliwość zapisania się do newslettera itp.
% ! [x] Określ w jaki sposób będziesz się komunikować z klientami w Internecie
% ! [x] Przemyśl koncepcję swojej strony i co dzięki niej klient będzie mógł zrobić

\subsubsection{Strona internetowa}

% Ogólnie
Stworzeniem oraz utrzymaniem strony internetowej zajmie się zespół ds. marketingu i sprzedaży internetowej.
Jej główną funkcjonalnością będzie sprzedaż produktu za pomocą platformy e-commerce.

\paragraph{Utworzenie}

% Rejestracja domeny .com
% Rejestracja hostingu
Zostanie przeprowadzony proces rejestracji domeny \textit{.com}, której wybór podkreśli globalny charakter działalności firmy.
Zarejestrujemy hosting typu \textit{VPS}, charakteryzujący się dużą wydajnością i pełna funkcjonalnością serwerów dedykowanych.

% Projekt graficzny
Kluczowym elementem będzie przyjazny projekt od strony wizualnej i użytkowej, za którego wykonanie odpowiedzialny będzie nasz wewnętrzny zespół informatyczny.

% E-commerce
Bazą sklepu internetowego będzie otwarte oprogramowanie Prestashop.
Jego główną zaletą jest duże wsparcie techniczne oraz prostota instalacji i użytkowania.
Nasi eksperci zadbają o odpowiednie zastosowanie najnowszych technologii i technik szyfrowania w celu zapewnienia bezpieczeństwa i niezawodności przebiegu procesu przeprowadzania transakcji.

% Uzupełnienie treści
Końcowym elementem stworzenia strony będzie uzupełnienie treści, tj. organizacja strony pod kątem zawartości, napisanie artykułów, itp.

\paragraph{Utrzymanie}

% Odnowienie domeny
% Odnowienie hostingu
Co miesiąc uiszczeniu będzie podlegać opłata za odnowienie domeny oraz hostingu.

% Administracja
Po utworzeniu strony, zespół zajmie się jej utrzymaniem i naprawą ewentualnych błędów.

% Wsparcie techniczne
Na stronie udostępniony zostanie również adres mailowy wsparcia technicznego produktu.
Klienci będą mogli pisać na niego w przypadku pytań lub wad funkcjonowaniu biblioteki sampli.

\subsubsection{Newsletter}

Jednym z elementów kontaktu z klientem będzie newsletter.
Zapisanie się do niego nastąpi automatycznie przy zakupie produktu ze strony internetowej.
Udostępniona zostanie jednak opcja wypisania się poprzez kliknięcie odpowiedniego linku w mailach promocyjnych.
Jest to standardowo rozpowszechniona praktyka.

Zapisanie się możliwe będzie także bezpośrednio z poziomu strony internetowej, jednakże będzie opcjonalne.

Główną zawartością maili promocyjnych będą informacje o promocjach, kolejnych produktach, itp.
Nie planujemy dużej liczby wysyłanych wiadomości z ograniczeniem do tych niezbędnych -- informacyjnych.

\subsubsection{Media społecznościowe}

Głównym elementem strategii marketingowej jest obecność w mediach społecznościowych, tj. na Facebooku, Instagramie i YouTubie.

Facebook będzie służył za główny kanał komunikacyjny z klientami.
Przemawiające za nim korzyści to wygoda użytkowania oraz prostota bezpośredniej komunikacji poprzez wiadomości.

% ! [x] Innym rozsyłanie biuletynów, newslettera albo utworzenie na swojej stronie internetowej forum, na którym klienci będą wymieniali opinie
% ! [x] To ostatnie jest dość ryzykowne, ponieważ mogą to być opinie negatywne
% ! [x] Dobrze jest w takim przypadku stworzyć strategię reagowania na nie (ignorowanie negatywnych opinii nie jest korzystnym wyjściem).

Zostanie założona oficjalna grupa użytkowników produktu.
Klienci będą mogli wymieniać się między sobą opiniami oraz komentarzami.
Zawartość grupy będzie moderowana, jednakże uważamy, że należy zezwolić kupującym na wolność słowa.
Negatywne opinie oraz pojawiające się w programie błędy będą dla nas cenną informacją przydatną do ulepszenia produktu.
Planujemy wydawać aktualizacje oprogramowania, więc cenimy sobie szczerość użytkowników.

Instagram będzie medium czysto promocyjnym.
Na profilu będziemy zamieszczać filmy i zdjęcia udostępniane przez użytkowników korzystających z produktu.
Można stwierdzić, że w pewien sposób będzie to strona współtworzona przez użytkowników.

YouTube przyda się jako miejsce, w którym umieścimy filmy instruktażowe -- o tym, jak zainstalować bibliotekę sampli, jak z niej korzystać, itp.

% ! [x] Pozycjonowanie obejmuje m.in. zgłoszenie swojej strony do wyszukiwarek
% ! [x] Obecność w nich i zajmowanie wysokiej pozycji na wyszukiwane słowa kluczowe jest ważna, żeby zapewnić stronie internetowej oglądalność
% ! [x] W tym miejscu możesz wymienić te wyszukiwarki (np. yahoo.com, google.pl, pl.ask.com)

\subsubsection{Pozycjonowanie}

% TODO: może dać to zewnętrznej firmie
Odpowiedzialni za to pracownicy pod przewodnictwem dyrektora ds. marketingu zgłoszą stronę internetową do najpopularniejszych wyszukiwarek: Google, Bing i Yahoo.
Zajmą się m.in. doborem słów kluczowych dobrze opisujących ofertę, weryfikacją strony i jej optymalizacją pod kątem treści oraz analizowaniem statystyk.

Wstępny plan zakłada skupienie się na uzyskaniu możliwie wysokiej pozycji w wyszukiwarkach przez pierwsze pół roku obecności produktu na rynku.
Analizując tendencje branżowe, bardziej istotne dla popularności produktu będzie skupienie się na obecności w mediach społecznościowych.
W pierwszym okresie, nastawionym na rozpoczęcie sprzedaży, chcemy jednak zwiększyć szanse i zapewnić dobry dostęp do produktu dla wszystkich osób.
Po upływie pół roku nastąpi weryfikacja przyjętej strategii.

% ! [x] Opisz brane pod uwagę sposoby reklamy
% ! [x] Zdecyduj gdzie chcesz je wykupić i dlaczego właśnie tam, jaką przybiorą formę
% + [ ] Jeśli wybierzesz reklamę w radiu, podaj jej długość, częstotliwość ukazywania się
% + [ ] Jeśli zdecydujesz się wziąć udział w targach/ imprezach/ seminariach, napisz kiedy się odbywają, jaką noszą nazwę
% ! [x] Stwórz harmonogram wykupowania reklam
% - [ ] Wybrane metody promocji i reklamy:
%   ! [x] Klasyczna reklama - ogłoszenia w prasie, radiu, telewizji, kinie, Internecie
%   - [ ] Marketing bezpośredni - mailing, rozdawanie ulotek na ulicy
%   - [ ] Public relations - artykuły o produktach firmy napisane przez pracownika lub wynajętego dziennikarza
%   - [ ] Wystawy, targi
%   - [ ] Wizyty u klientów

\subsubsection{Reklamy}

Mimo obecności stron w Internecie, należy zainwestować osobno w strategię reklamową.
Planujemy utrzymać reklamy na Facebooku przez pierwsze pół roku, natomiast marketing na YouTubie -- na czas nieokreślony.

\paragraph{Facebook}

Uważamy, że sponsorowanie postów na Facebooku będzie dla nas najbardziej skuteczną formą reklamy.
Interfejs Facebooka pozwoli nam na wyświetlanie reklam dla osób ze zdefiniowanej grupy docelowej.
Materiały promocyjne można również bezpośrednio integrować w zawartość postów na naszym profilu.

Koszt reklamy na Facebooku przypada w granicach 0,40--4,00 zł za jedną konwersję \cite{facebook}.
Zakładając najwyższą cenę -- 4 zł -- i liczbę interesujących nas miesięcznych konwersji -- 2000 --, przeliczających się na liczbę polubień na stronie, koszt reklamy wyniesie nas 8000 zł na miesiąc.

\paragraph{YouTube}

Umieszczane za pomocą interfejsu Google Ads, filmy promocyjne w serwisie YouTube pozwolą nam na zaprezentowanie oferowanych brzmień sampli perkusyjnych oraz interfejsu programu i pochlebnych opinii klientów.
Jest to też dobra okazja na wyjście z utartych schematów i wykonanie kreatywnej reklamy skierowanej do muzyków -- osób kreatywnych.

Reklama na YouTubie będzie funkcjonować jako suplement reklamy na Facebooku.
W związu z tym, przeznaczymy na nią mniej pieniędzy.

% \paragraph{Google Ads ??? zobacz czy może nie zrezygnować z tego, youtube jest przez adsy robione}

% Podobnie, jak w przypadku pozycjonowania, przez pierwsze pół roku zainwestujemy w dodatkową reklamę przy pomocy banerów wyświetlanych w usłudze Google Ads.
% Planujemy jednak utrzymać reklamę na Facebooku i YouTubie przez dłuższy okres.

\paragraph{Pozostałe formy}

Nie bierzemy pod uwagę klasycznych form reklamy, tj. telewizji, radia i prasy.
Patrząc po branży muzycznej, obecność marketingowa w tych mediach nie jest konieczna.
Grupa docelowa ma tendencję do skupiania się na ,,nowych'' mediach, przez co umieszczanie reklam w tradycyjnych kanałach nie byłoby skuteczne.

% - [ ] Działania które wyszczególnisz, muszą mieć umocowanie w budżecie, a także przedstawiać realne rezultaty
% - [ ] Napisz gdzie wydasz pieniądze, ile, czemu w ten sposób i czego się spodziewasz po takiej kampanii
% - [ ] Koszty należy wpisać w plan finansowy
\begin{table}[h!]
 \begin{center}
 \makebox[\linewidth]{%
 \begin{tabular}{llrr}
 \toprule
     \multicolumn{2}{l}{\multirow{2,5}{*}{\textbf{Działanie}}} &
        \multicolumn{2}{c}{\textbf{Koszt}} \\
 \cmidrule(lr){3-4}
     & & \multicolumn{1}{c}{\textit{Jednorazowy}} & \multicolumn{1}{c}{\textit{Cykliczny}} \\
 \midrule
     % Utworzenie strony internetowej
     \multirow{5}{*}{\makecell[l]{Utworzenie\\strony\\internetowej}}
     & Rejestracja domeny \textit{.com}        & 49,00 zł~\cite{domena}          & --- \\
     & Rejestracja hostingu                    & \$26,96~\cite{hosting}          & --- \\
     & Projekt graficzny                       & 10~000 zł~\cite{projekt-strony} & --- \\
     & Wdrożenie systemu e-commerce Prestashop & 15~000~zł~\cite{e-commerce}     & --- \\
     & Uzupełnienie treści                     & 5~000~zł                        & --- \\
 \midrule
    % Utrzymanie strony internetowej
     \multirow{5}{*}{\makecell[l]{Utrzymanie\\strony\\internetowej}}
     & Odnowienie domeny   & --- & 99,99 zł/mies.~\cite{domena} \\
     & Odnowienie hostingu & --- & \$29,15/mies.\cite{hosting}  \\
     & Administracja       & --- & 3~000~zł/mies.               \\
     & Newsletter          & --- & 500~zł/mies.                 \\
     & Wsparcie techniczne & --- & 2~000~zł/mies.               \\
 \midrule
    % Media społecznościowe
     \multirow{3}{*}{\makecell[l]{Media\\społecznościowe}}
     & Utworzenie profili & 500~zł & ---            \\
     & Utrzymanie profili & ---    & 1~500~zł/mies. \\
     & Moderowanie grup   & ---    & 500~zł/mies.   \\
 \midrule
    % Wyszukiwarki
     \multirow{1}{*}{Wyszukiwarki}
     & Pozycjonowanie & --- & 4~500~zł/mies.~\cite{pozycjonowanie} \\
 \midrule
    % Reklamy
     \multirow{2}{*}{Reklamy}
     & Facebook   & --- & 8~000~zł/mies.~\cite{facebook} \\
     & YouTube    & --- & 2~000~zł/mies.~\cite{youtube}  \\
     % & Google Ads & --- & ---                            \\
 \bottomrule
 \end{tabular}}
 \caption{Koszty prowadzenia marketingu i sprzedaży internetowej}
 \label{table:marketing-sprzedaz}
 \end{center}
\end{table}

\subsection{Pozyskiwanie klientów}

% ! [x] Pozyskać klientów można w rozmaity sposób, z wykorzystaniem wielu metod
% ! [x] To kiedy planujesz dotrzeć do klientów jest równie ważne jak to jakimi metodami to zrobisz
% ! [x] Warto również uwzględnić darmowe sposoby reklamy
% ! [x] Z pomocą przyjdzie tutaj Internet
% + [ ] Inny sposób to wymiana linków z firmami, które sprzedają towar uzupełniający ofertę firmy (a więc nie ma konfliktu interesów)

Planujemy dotrzeć do naszych potencjalnych klientów od początku okresu sprzedaży.
Możemy wykorzystać do tego kilka metod.

\subsubsection{Mike Malyan}

% ! [x] Obecność w portalach społecznościowych (np. Facebook, Twitter) nic nie kosztuje, a może zachęcić klientów do zakupu

Naszym głównym wspólnikiem jest brytyjski muzyk, perkusista Mike Malyan.
Przygotowywany produkt to wirtualny instrument bazujący na samplach perkusyjnych stworzonych przy pomocy prawdziwego zestawu perkusyjnego należącego do Malyana.

Artysta przez lata zbudował bazę fanów jego twórczości, co potwierdzają np. dane z serwisów społecznościowych.
Możemy wykorzystać tę popularność poprzez umieszczanie materiałów reklamowych na jego profilach.
Pozwoli nam to znacząco zmniejszyć koszty przy jednoczesnym potencjalnym dużym zysku reklamowym.
Rozważamy tworzenie filmów promocyjnych z udziałem Malyana, w których muzyk zachęci potencjalnych kupujących oraz zademonstruje produkt sygnowany jego nazwiskiem.

\subsubsection{Grupy na Facebooku}

Facebook pozwala na tworzenie grup, w których ludzie zrzeszają się np. w ramach wspólnych zainteresowań.
Obecnie istnieją grupy zrzeszające m.in. miłośników twórczości Malyana, produkcji muzycznej, nowoczesnych technologii w studiach, itp.
Umieścimy w tych grupach posty reklamujące nasz produkt.

\subsubsection{Artykuły o produkcie}

% ! [x] Istotne są działania związane z PR
% ! [x] Jeśli produkt lub usługa jest czymś wyjątkowym, co wzbudza ciekawość możesz wysyłać informację prasową do mediów
% ! [x] Wydawcy stron chętnie zamieszczają tego rodzaju newsy
% ! [x] Można również zamieszczać artykuły/posty na wielu stronach funkcjonujących w Internecie

Istnieje obecnie wiele portali internetowych zajmujących się tematyką produkcji muzycznej.
Mają swoje strony internetowe oraz profile w mediach społecznościowych.
Spróbujemy napisać do nich propozycję udostępnienia naszego produktu.
Nasz program będzie pierwszą sygnaturą Malyana oraz jako jedyny na rynku w tym segmencie użyje sztucznej inteligencji do tworzenia realistycznie brzmiących ścieżek.
Sądzimy, że zainteresuje to redaktorów portali i napiszą o tym, ponieważ przyniesie im to zwiększoną liczbę czytelników.

\subsection{Utrzymywanie klientów}

% ! [x] Utrzymanie klienta jest o wiele tańsze niż zdobycie go
% ! [x] Można to zrobić poprzez np. stworzenie listy klientów zawierającej ich adresy
% ! [x] Następnie zaplanować dla nich jakąś korzyść, utrzymywać z nimi kontakt, pozostać w ich świadomości
% + [ ] Jednym ze sposobów są programy lojalnościowe lub preferencyjne

Po zainwestowaniu w marketing i zdobyciu bazy klientów, zależy nam na ich utrzymaniu.
Znając listę obecnych kupujących zapisanych do newslettera oraz tych, którzy obserwują nasze profile w mediach społecznościowych, będziemy mogli utrzymać z nimi stały kontakt.

Poprzez te kanały zaproponujemy program poleceń.
Każdy klient, który zdążył już zakupić produkt, będzie mógł dać swojemu znajomemu specjalny odnośnik upoważniający do zniżki dla nowego klienta.

Istnienie naszej grupy na Facebooku będzie działało zachęcająco dla istniejących klientów.
Przyczyni się do tego możliwość wypowiedzenia się a jednocześnie zapoznania z opiniami i materiałami innych.
Zachęcimy klientów do postowania nagrań muzycznych, w których użyli naszej biblioteki sampli.
Będzie to inspiracja dla innych użytkowników, która może zwiększyć ich lojalność do naszego produktu.

Regularna aktywność w mediach społecznościowych sprawi, że pozostaniemy w świadomości klientów.

\subsection{Struktura przychodów}

% ! [x] Celem działalności jest zysk
% ! [x] Wygenerujesz go dzięki sprzedaży towarów lub usług

Źródłem przychodów tej działalności będzie sprzedaż biblioteki sampli Mike'a Malyana.

% ! [x] Dzięki badaniom rynkowym dowiesz się ile powinna wynosić
Nasz produkt jest najbardziej podobny do instrumentów oferowanych przez \textit{GetGood Drums}.
Średnia cena produktu tej firmy według tablicy \ref{table:competition} to \$93.

% ! [x] Zdecyduj po jakiej cenie będą sprzedawane produkty lub usługi
Cena naszego produktu będzie niższa -- \emph{\productpricezl}\footnote{ok. \$79 (kurs z 9 listopada 2020)}.
% ! [x] Cena jaką ustalisz musi być na tyle wysoka, aby pokryć wydatki i zapewnić zysk, a jednocześnie tak dobrana, aby klient był gotowy ją zapłacić (niekoniecznie oznacza to ceną najniższą)
Uważamy, że jest to dobry balans cenowy zapewniający mniejszy wydatek dla klienta niż konkurencja przy jednoczesnym maksymalizowaniu zysku i pokryciu wydatków.

% ! [x] Nowe firmy preferują najczęściej mniej klientów i wyższą cenę
Sądzimy, że celowanie w wyższe ceny znacząco obniżyłoby liczbę potencjalnych klientów i zainteresowanie produktem.
\textit{GetGood Drums} ma podobną grupę docelową, więc nasza niższa cena gwarantuje, że potencjalni klienci będą mogli zakupić nasz produkt.

% - [ ] Jednostką sprzedaży jest pojedynczy produkt lub usługa, np. godzina pracy, rzecz którą sprzedajesz
% - [ ] Podaj strukturę cen i plan uzyskania przychodu
% - [ ] Jeśli planujesz sprzedawać przez Internet, wyjaśnij jak zamienisz ruch na swojej stronie na przychody
% - [ ] Ważny jest przyjazny użytkownikom mechanizm sprzedaży
% + [ ] Innym sposobem na zarabianie na stronie jest pozyskanie sponsora lub uczestnictwo w programach partnerskich

% ? <Cena produktu>

\subsection{Strategia sprzedaży}

% ! [x] Opisz w jaki sposób będziesz prowadzić sprzedaż
% ! [x] Sposoby dystrybucji występują pośrednie i bezpośrednie, np.
%   + [ ] Detaliści zewnętrzni
%   + [ ] Przedstawiciele zewnętrzni
%   + [ ] Licencja franchisingowa
%   + [ ] Hurtownicy
%   + [ ] Sklepy firmowe
%   + [ ] Własna sieć sprzedaży bezpośredniej
%   + [ ] Mailing
%   ! [x] Internet

Sprzedaż będzie prowadzona wyłącznie poprzez Internet.

% ! [x] Należy opisać jak będzie sprzedawany produkt lub usługa, kto będzie to robił
W celu zakupienia towaru za pomocą strony internetowej, klient zostanie automatycznie przeprowadzony przez standardową procedurę:
\begin{enumerate}
    \item założenia konta i uzupełnienia wymaganych danych osobowych,
    \item dodania produktu do koszyka,
    \item uiszczenia opłaty,
    \item otrzymania drogą mailową odnośników do pobrania zakupionego oprogramowania.
\end{enumerate}

% ! [x] Strategia sprzedaży powinna być zsynchronizowana ze strategią marketingową
Strategia sprzedaży będzie zsynchronizowana ze strategią marketingową.

% <pozyskanie klientów>
Po stworzeniu strony internetowej, przez pierwsze pół roku dystrybucja będzie wspierana przez pozycjonowanie w wyszukiwarkach internetowych.
Mike Malyan zapoczątkuje reklamę produktu poprzez swoje profile w mediach społecznościowych.
Pomogą też artykuły na portalach muzycznych i ich profilach.
Umieszczone zostaną także posty w grupach na Facebooku oraz na profilach firmy w mediach społecznościowych.

% <utrzymanie klientów>
Przez pełen okres sprzedaży będzie funkcjonował rozsyłany drogą mailową newsletter.
Poza tym, produkt będzie obecny w mediach społecznościowych oraz reklamach.

% ! [x] Strategię sprzedaży można zaprezentować w formie tabelki, w której będą przedstawione prognozy sprzedaży na kolejne lata
\begin{table}[h!]
 \begin{center}
 \makebox[\linewidth]{%
 \begin{tabular}{lrrrr}
 \toprule
     \multicolumn{1}{c}{\multirow{2,5}{*}{\textbf{Okres}}} & \multicolumn{2}{c}{\textbf{Sprzedane egzemplarze}} & \multicolumn{2}{c}{\textbf{Zysk}} \\
 \cmidrule(lr){2-3}
 \cmidrule(lr){4-5}
     & \multicolumn{1}{c}{\textit{Okresowo}} & \multicolumn{1}{c}{\textit{Łącznie}} & \multicolumn{1}{c}{\textit{Okresowo}} & \multicolumn{1}{c}{\textit{Łącznie}} \\
 \midrule
     \multirow{2}{*}{%
     \begin{tabular}{@{}ll@{}}
         \multirow{2}{*}{1. rok} & 1. połowa \\
          & 2. połowa \\
     \end{tabular}
     } & 12 000 & 12 000 & 3 588 000 zł & 3 588 000 zł \\
     & 6 000 & 18 000 & 1 794 000 zł & 5 382 000 zł \\
 \midrule
     2. rok & 3 000 & 21 000 & 897 000 zł & 6 279 000 zł \\
 \midrule
     3. rok & 1 500 & 22 500 & 448 500 zł & 6 727 500 zł \\
 \midrule
     4. rok & 750 & 23 250 & 224 250 zł & 6 951 750 zł \\
 \bottomrule
 \end{tabular}}
 \caption{Prognoza sprzedaży na pierwsze 4 lata}
 \label{table:sales}
 \end{center}
\end{table}

% + [ ] Jeśli chcesz skorzystać z własnego personelu, a produkt jest czymś skomplikowanym technologicznie, możesz rozważyć przeprowadzenie specjalnych szkoleń związanych z jego obsługą
% ! [x] Opisz w jaki sposób sprzedawcy będą utrzymywać kontakt z klientami, jakie umiejętności powinni posiadać, w jaki sposób ich zrekrutujesz
% + [ ] Jeśli sam/a będziesz sprzedawcą, powinieneś/powinnaś określić swoje doświadczenie w sprzedaży
% + [ ] Napisz co wiesz o sprzedawaniu tego produktu, usługi
% ! [x] Podaj płace i prowizje sprzedawców, świadczenia i zachęty
% + [ ] Uwzględnij w działaniach kontakty telefoniczne i reklamę bezpośrednią, opisz szczegóły w planie
Sprzedaż będzie przeprowadzana bez udziału sprzedawców.
Klient będzie mógł zakupić produkt samodzielnie.
Najczęściej spotykane pytania odnośnie produktu i przebiegu procesu kupna umieścimy w sekcji FAQ (Frequently Asked Questions).
W przypadku wystąpienia niespotkanych wcześniej problemów, klient będzie mógł napisać na specjalny mail do pomocy technicznej.
Odpowiadanie na pytania drogą mailową, za co będą odpowiedzialni pracownicy zespołu do spraw marketingu i sprzedaży internetowej, wpisane jest w koszty utrzymania strony.

\section{System biznesowy i organizacja}

% - [ ] Należy przemyśleć w jaki sposób dokonamy podziału prac, aby nasze przedsięwzięcie było realizowane najwydajniej
% ! [x] Wybierz działania, jakie będą wykonywane w Twojej firmie i pogrupuje w bloki funkcyjne, np.
%   + [ ] badania i rozwój
%   ! [x] produkcja
%   ! [x] marketing
%   ! [x] sprzedaż
%   ! [x] obsługa/serwis

% ! [x] Niezwykle ważne jest stworzenie prostej i funkcjonalnej struktury organizacyjnej
% - [ ] Należy być przygotowanym do elastycznej reorganizacji firmy, przynajmniej w pierwszych kilku latach działalności
% ! [x] Zdecyduj kto za co odpowiada
% ! [x] Prosta struktura ułatwi sporządzenie opisów stanowisk i zatrudnienia na nie odpowiednich osób
Prosta i funkcjonalna struktura organizacyjna jest kluczowa dla rozwoju firmy.
W tablicy \ref{table:system-biznesowy} umieściliśmy podział działań firmy ze względu na bloki funkcyjne.

% ! [x] Działania wykraczające poza główną domenę działalności firmy powinny zostać zlecone podwykonawcom
% ! [x] Działania dla firmy strategiczne należy zostawić pod bezpośrednią kontrolą zarządu firmy, ale np. księgowość, czy zarządzanie kadrami można zlecić komuś z zewnątrz
Większość strategicznych działań pozostawiamy pod bezpośrednią kontrolą zarządu firmy.
Działania, takie jak realizację nagrań sampli w profesjonalnym studio muzycznym, zlecimy podwykonawcom.

\begin{table}[h!]
 \begin{center}
 \makebox[\linewidth]{%
     \begin{tabular}{llll}
 \toprule
         \multicolumn{1}{c}{\multirow{2,5}{*}{\textbf{Blok funkcyjny}}} & \multicolumn{1}{c}{\multirow{2,5}{*}{\textbf{Działania}}} & \multicolumn{2}{c}{\textbf{Wykonawca}} \\
 \cmidrule(lr){3-4}
         & & \multicolumn{1}{c}{\textit{Firma}} & \multicolumn{1}{c}{\textit{Nadzór}} \\
 \midrule
     \makecell[l]{Realizacja\\nagrań} &
     \makecell[l]{%
        \tabitem przygotowanie studia \\
        \tabitem nagranie sampli perkusyjnych \\
        \tabitem nagranie utworów w wykonaniu \\\phantom{\tabitem}Mike'a Malyana \\
        \tabitem wstępna obróbka nagrań \\
     } &
     Zewnętrzna &
     \makecell[l]{%
         \tabitem dyr. ds. kontroli jakości \\
    }\\
 \midrule
     \makecell[l]{Obróbka\\nagrań} &
     \makecell[l]{%
        \tabitem edycja nagrań \\
        \tabitem eksport sampli perkusyjnych \\\phantom{\tabitem}do odpowiedniego formatu \\
        \tabitem organizacja plików nagrań \\
        \tabitem przygotowanie nagranych \\\phantom{\tabitem}próbek gry muzyka \\
     } &
     \textit{\nazwafirmy{}} &
     \makecell[l]{%
         \tabitem dyr. ds. kontroli jakości \\
         \tabitem dyr. ds. produkcji \\
            \phantom{\tabitem}muzycznej \\
     } \\
 \midrule
     \makecell[l]{Inżynieria\\oprogramowania} &
     \makecell[l]{%
        \tabitem nauka sieci neuronowych \\
        \tabitem projekt i implementacja \\\phantom{\tabitem}wtyczki VSTi \\
        \tabitem testowanie oprogramowania \\
        \tabitem aktualizacje i wsparcie techniczne \\
     } &
     \textit{\nazwafirmy{}} &
     \makecell[l]{%
        \tabitem dyr. ds. kontroli jakości \\
         \tabitem dyr. ds. inżynierii \\
            \phantom{\tabitem}oprogramowania \\
     } \\
 \midrule
     \makecell[l]{Marketing\\i sprzedaż\\internetowa} &
     \makecell[l]{%
        \tabitem utworzenie i utrzymanie strony \\\phantom{\tabitem}internetowej \\
        \tabitem wdrożenie platformy e-commerce \\
        \tabitem media społecznościowe \\
        \tabitem reklamy i pozycjonowanie \\
        \tabitem zarządzanie procesem sprzedaży \\
     } &
     \textit{\nazwafirmy{}} &
     \makecell[l]{%
         \tabitem dyr. ds. marketingu \\\phantom{\tabitem}i sprzedaży internetowej \\
    } \\
 \bottomrule
 \end{tabular}}
 \caption{Działania realizowane w trakcie inwestycji, podzielone na bloki funkcyjne}
 \label{table:system-biznesowy}
 \end{center}
\end{table}

\section{Harmonogram realizacji procesów}

% ! [x] Poszczególne zadania należy pogrupować w pakiety robocze
% ! [x] Każdy z nich musi mieć kolejne etapy, które kończą się osiągnięciem zamierzonego celu
% - [ ] Lepiej planować pesymistycznie niż optymistycznie

\subsection{Realizacja nagrań}

\begin{itemize}
    \item znalezienie i zatrudnienie inżyniera oraz producenta muzycznego
    \item wynajęcie studia muzycznego w Wielkiej Brytanii
    \item transport zestawu perkusyjnego do studia
    \item nagranie sampli perkusyjnych
    \item nagranie utworów w wykonaniu Mike'a Malyana
    \item wstępna obróbka nagrań
    \item przesłanie wykonanych nagrań do firmy \textit{\nazwafirmy}
\end{itemize}

\subsection{Obróbka nagrań}

\begin{itemize}
    \item edycja nagrań pod kątem barwy i dynamiki
    \item organizacja plików nagrań
    \item eksport sampli perkusyjnych do odpowiedniego formatu
    \item przygotowanie nagranych próbek gry muzyka
\end{itemize}

\subsection{Stworzenie wtyczki VSTi}

\begin{itemize}
    \item wykonanie projektu oprogramowania
    \item nauka sieci neuronowych na próbkach gry muzyka
    \item implementacja wtyczki VSTi
    \item testowanie oprogramowania i poprawki błędów
\end{itemize}

\subsection{Marketing i sprzedaż internetowa}

\begin{itemize}
    \item strona internetowa \begin{itemize}
        \item utworzenie \textit{(jednorazowo)} \begin{itemize}
            \item rejestracja domeny i hostingu
            \item wykonanie projektu graficznego (front end)
            \item wykonanie zaplecza technicznego (back end)
            \item wdrożenie systemu e-commerce
            \item uzupełnienie treści \end{itemize}
        \item utrzymanie \textit{(cyklicznie)} \begin{itemize}
            \item odnowienie domeny i hostingu
            \item administracja
            \item zarządzanie newsletterem
            \item wsparcie techniczne \end{itemize} \end{itemize}
    \item newsletter \begin{itemize}
        \item cykliczne tworzenie i wysyłanie maili promocyjnych \end{itemize}
    \item media społecznościowe \begin{itemize}
        \item założenie \textit{(jednorazowo)} \begin{itemize}
            \item utworzenie profili na Facebooku, Instagramie i YouTubie
            \item założenie grupy fanów na Facebooku \end{itemize}
        \item utrzymanie \textit{(cyklicznie)} \begin{itemize}
            \item tworzenie zawartości i regularne postowanie na stronach produktu
            \item umieszczanie materiałów promocyjnych na stronie Mike'a Malyana
            \item oferta zniżkowa poprzez program poleceń dla lojalnych klientów
            \item moderowanie profili i grup \end{itemize} \end{itemize}
    \item pozycjonowanie \begin{itemize}
        \item zgłoszenie strony do wyszukiwarek
        \item dobór słów kluczowych, weryfikacja strony i optymalizacja pod kątem treści
        \item cykliczna analiza statystyk \textit{(pierwsze pół roku)} \end{itemize}
    \item reklamy \begin{itemize}
        \item utworzenie reklam skierowanych pod daną platformę
        \item umieszczenie reklam na każdej docelowej platformie
        \item analiza skuteczności kampanii \end{itemize}
    \item nawiązanie kontaktu z portalami muzycznymi w celu promocji poprzez dedykowane artykuły
\end{itemize}

\subsection{Wsparcie techniczne}

\begin{itemize}
    \item utworzenie na stronie internetowej sekcji często zadawanych pytań \textit{(FAQ)}
    \item udostępnienie adresów mailowych i formularzy kontaktowych wsparcia technicznego związanego z produktem oraz procesem kupna
    \item regularne odpowiedzi na pytania użytkowników i potencjalnych klientów
\end{itemize}

\subsection{Aktualizacje}

\begin{itemize}
    \item analiza opinii i uwag klientów
    \item wdrażanie poprawek
    \item optymalizacje działania aplikacji
\end{itemize}

\section{Możliwości i zagrożenia}

% ! [x] Źródłem zagrożenia może być sama firma jak też czynniki zewnętrzne
% ! [x] Oceny zagrożeń można dokonać przez stworzenie scenariuszy rozwoju firmy. W biznesplanie powinny być trzy:
%   ! [x] scenariusz bazowy - najbardziej prawdopodobny
%   ! [x] scenariusz optymistyczny
%   ! [x] scenariusz pesymistyczny - wszystkie zagrożenia jakie się pojawią

Jesteśmy świadomi faktu, że firma może być narażona zarówno na czynniki zewnętrzne, jak również wewnętrzne.
Rozpatrujemy trzy możliwe scenariusze przebiegu produkcji oraz sprzedaży produktu.

\subsection{Scenariusz bazowy}

\begin{itemize}
    \item bezproblemowy przebieg działań strategicznych: realizacji i obróbki nagrań, stworzenia wtyczki VSTi, uruchomienia kanałów marketingowych oraz stworzenia strony internetowej z platformą e-commerce
    \item sukces kampanii marketingowej -- pozyskanie zakładanej liczby klientów
    \item zdobycie popularności produktu wśród grupy docelowej i w branży producenckiej
    \item pokrycie wstępnych wydatków i uzyskanie zakładanych zysków
    \item minimalna liczba krytycznych błędów aplikacji -- niska potrzeba pilnych aktualizacji
\end{itemize}

\subsection{Scenariusz optymistyczny}

\begin{itemize}
    \item pozyskanie większej niż zakładano liczby klientów
    \item zdobycie uznania znanych producentów w branży i wynikająca z tego darmowa reklama oraz większy zasięg
    \item popularność firmy i produktu poza grupą docelową, większa rozpoznawalność w branży muzycznej
    \item uzyskanie większych zysków niż w planie finansowym
    \item oprogramowanie bezawaryjne, niewymagające poprawek błędów
\end{itemize}

\subsection{Scenariusz pesymistyczny}

\begin{itemize}
    \item problemy w trakcie działań strategicznych: zwiększone koszty, opóźnienia w harmonogramie, nieprzewidziane problemy
    \item pozyskanie niewystarczającej liczby klientów do pokrycia kosztów wstępnych produkcji
    \item nieskuteczność kampanii marketingowej
    \item negatywne recenzje produktu przez znawców i klientów
    \item straty finansowe
    \item wadliwe oprogramowanie wymagające wielu kosztownych poprawek
    \item nieporozumienia wewnątrz firmy dotyczące zarządzania procesami -- potencjalny rozpad stabilnej struktury zarządu oraz upadłość firmy
\end{itemize}

\section{Plany finansowe}

% ! [x] Zastanów się ile pieniędzy trzeba na założenie i prowadzenie firmy
% - [ ] Ile powinniśmy mieć aby firma mogła regulować wydatki na bieżąco?
% ! [x] Jak uzyskać fundusze?

% - [ ] Elementy planu finansowego
%   ! [x] bilans
%   - [ ] założenia
%   ! [x] rachunek zysków i strat
%   - [ ] rachunek przepływów pieniężnych

\subsection{Analiza}

W tablicy \ref{table:plan-finansowy} przedstawiliśmy przybliżony miesięczny plan finansowy na pierwsze 4 lata.
Jedynym źródłem przychodu będzie internetowa sprzedaż produktu (tablica \ref{table:sales}).
Do głównych cyklicznych wydatków należą: działalność internetowa (marketing i sprzedaż) oraz pensje.

\paragraph{Działalność internetowa}

Okresowe wydatki na działalność internetową (tablica \ref{table:marketing-sprzedaz}) przez pierwsze pół roku uwzględniają reklamy na Facebooku oraz opłaty za pozycjonowanie.
Przez kolejne miesiące, koszt za ten dział inwestycji jest niezmienny.

\paragraph{Pensje}

Na pensje pracownicze będą składać się miesięczne wynagrodzenia dla dyrektorów działów oraz dla pracowników.
Planujemy zatrudnić 4 pracowników na okres nieokreślony.
Ich głównym zadaniem będzie administracja infrastrukturą sprzedaży oraz nadzorowanie i poprawki programu.
Kolejnych 6 pracowników zatrudnimy na okres jednego roku.
Będą odpowiedzialni za wdrożenie technologii oraz początkowy nadzór nad jej funkcjonowaniem.
Decyzje o formie zatrudnienia motywujemy założeniem o malejącym w czasie zainteresowaniem produktem i zmniejszającymi się zyskami.
Nasz produkt będzie potrzebował z biegiem czasu mniej zainwestowanych godzin roboczych -- tylko na konserwację, stąd zmniejszone zapotrzebowanie na etaty.

\paragraph{Koszty wstępne}

Koszty związane z wytworzeniem produktu (tablica \ref{table:jednorazowe}) zostaną pokryte z wkładu udziałowców firmy.

\paragraph{Opłaty stałe}

Nie uwzględnialiśmy opłat stałych, np. czynszu za biuro czy opłat za prąd, ponieważ cały zespół pracuje zdalnie.

\subsection{Podsumowanie}

Patrząc na okresowy bilans zysków na pierwsze 4 lata działalności (tablica \ref{table:bilans-zyskow}) zauważamy, że główna część zysku -- ok. 4 mln zł -- zostanie pozyskana w trakcie pierwszego roku.

Jeśli zainteresowanie produktem będzie duże na początku i malejące z czasem, inwestycja będzie przynosić zyski przez pierwsze 2 lata.
Lata 3 i 4, mimo zmniejszenia wydatków, przyniosą straty.

Po pierwszych 2 latach działalności zostanie podjęta decyzja, czy produkt wart jest wsparcia na kolejne lata.
W optymistycznym przypadku stałego lub rosnącego zainteresowania produktem, działalność nie przyniesie strat w latach 3 i 4.
Jeśli firma wypuści w okresie drugiego roku inny produkt, kwestia wsparcia mimo strat pozostanie przedyskutowana w celu zatrzymania obecnych klientów, oferując im ciągłe wsparcie produktu.
Straty w latach 3 i 4 mogą wtedy zostać potraktowane jako rodzaj marketingu kolejnego produktu i utrzymanie dobrego wizerunku firmy.

Podsumowując, biblioteka sampli perkusyjnych sygnowana nazwiskiem Mike'a Malyana przyniesie ok. 3,2 mln zł zysku przy przewidywanym wsparciu na 4 lata, oraz ok. 4,1 mln zysku, jeśli wsparcie potrwa 2 lata.

% - [ ] Zadaj sobie kilka pytań:
%   - [ ] Ile pieniędzy firma potrzebuje i w jakim okresie?
%   - [ ] Jakich zysków można się spodziewać po wprowadzeniu firmy na rynek?
%   - [ ] Na jakich założeniach oparte są prognozowane wyniki?

\begin{table}[h!]
 \begin{center}
 \makebox[\linewidth]{%
 \begin{tabular}{llr}
 \toprule
     \multicolumn{2}{l}{\textbf{Działania}} & \multicolumn{1}{c}{\textbf{Koszt}} \\
 \midrule
     \multirow{3}{*}{Realizacja} & Zatrudnienie inżyniera oraz producenta muzycznego & 20 000 zł \\
     & Wynajęcie studia muzycznego w Wielkiej Brytanii & 22 400 zł \\
     & Transport zestawu perkusyjnego & 1 000 zł \\
 \midrule
     \multirow{1}{*}{Obróbka} & Edycja i przygotowanie nagrań & 20 000 zł \\
 \midrule
     \multirow{4}{*}{Oprogramowanie} & Projekt & 10 000 zł \\
     & Nauka sztucznej inteligencji & 10 000 zł \\
     & Implementacja & 30 000 zł \\
     & Testowanie & 10 000 zł \\
 \midrule
     \multirow{1}{*}{Działalność internetowa} & Jednorazowe koszty \textsuperscript{\textit{(patrz: tablica \ref{table:marketing-sprzedaz})}} & 30 651 zł \\
 \midrule
     \multirow{1}{*}{Praca zdalna} & Wyposażenie pracowników & 66 000 zł \\
 \bottomrule
 \end{tabular}}
 \caption{Jednorazowe wydatki na wytworzenie produktu}
 \label{table:jednorazowe}
 \end{center}
\end{table}

\begin{table}[h!]
 \begin{center}
 \makebox[\linewidth]{%
 \begin{tabular}{llrrrr}
 \toprule
     \multicolumn{2}{c}{\multirow{2,5}{*}{\textbf{Okres}}} & \multicolumn{1}{c}{\textbf{Przychody}} & \multicolumn{2}{c}{\textbf{Wydatki}} & \multicolumn{1}{c}{\multirow{2,5}{*}{\textbf{Bilans}}} \\
 \cmidrule(lr){3-3}
 \cmidrule(lr){4-5}
     & & \multicolumn{1}{c}{\textit{Sprzedaż}} & \multicolumn{1}{c}{\textit{Działalność internetowa}} & \multicolumn{1}{c}{\textit{Pensje}} & \\
 \midrule
     \multirow{2,5}{*}{1. rok} & 1. połowa & 598 000 zł & 22 210,47 zł & 89 000 zł & 486 789,53 zł \\
 \cmidrule(l){3-6}
                               & 2. połowa & 299 000 zł & 9 710,47 zł  & 89 000 zł & 200 289,53 zł \\
 \midrule
     2. rok                    &           & 74 750 zł  & 9 710,47 zł  & 59 000 zł & 6 039,53 zł   \\
 \midrule
     3. rok                    &           & 37 375 zł  & 9 710,47 zł  & 59 000 zł & -31 335,47 zł \\
 \midrule
     4. rok                    &           & 18 688 zł  & 9 710,47 zł  & 59 000 zł & -50 022,97 zł \\
 \bottomrule
 \end{tabular}}
 \caption{Miesięczny plan finansowy na pierwsze 4 lata}
 \label{table:plan-finansowy}
 \end{center}
\end{table}

\begin{table}[h!]
 \begin{center}
 \makebox[\linewidth]{%
 \begin{tabular}{llrr}
 \toprule
     \multicolumn{2}{c}{\multirow{2,5}{*}{\textbf{Okres}}} & \multicolumn{2}{c}{\textbf{Bilans}} \\
 \cmidrule(lr){3-4}
     & & \multicolumn{1}{c}{\textit{Okresowo}} & \multicolumn{1}{c}{\textit{Łącznie}} \\
 \midrule
     \multirow{2,5}{*}{1. rok} & 1. połowa & 2 921 tys. zł & 2 921 tys. zł \\
 \cmidrule(l){3-4}
                               & 2. połowa & 1 202 tys. zł & 4 123 tys. zł \\
 \midrule
     2. rok                    &           & 72 tys. zł    & 4 195 tys. zł \\
 \midrule
     3. rok                    &           & -376 tys. zł  & 3 819 tys. zł \\
 \midrule
     4. rok                    &           & -600 tys. zł  & 3 219 tys. zł \\
 \bottomrule
 \end{tabular}}
 \caption{Okresowy bilans zysków na pierwsze 4 lata}
 \label{table:bilans-zyskow}
 \end{center}
\end{table}

\printbibliography

\end{document}
